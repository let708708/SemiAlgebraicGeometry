\documentclass[dvipdfmx]{jsarticle}


\usepackage{amsmath,amssymb}
\usepackage{amsthm}

\newcommand{\R}{\mathbb{R}}

\theoremstyle{definition}
\newtheorem{definition}{定義}[section]
\newtheorem{proposition}{命題}[section]
\newtheorem{lemma}{補題}[section]
\newtheorem{theorem}{定理}[section]
\newtheorem{corollary}{系}[section]
\newtheorem*{claim*}{主張}

\renewcommand{\proofname}{\textbf{証明}}



\begin{document}

\title{QE for RCF by CAD}
\author{富永 直弥}
\maketitle

\section{CADアルゴリズム}
CADとは, ユークリッド空間の分割で, 各分割の上で複数の多項式を符号不変にするものである. Collinsにより提唱されたCADアルゴリズムについてかく. 
\begin{definition}
$\R^n$の有限部分集合族$\mathfrak{D}$が, 

\begin{itemize}
	\item 任意の$D \in \mathfrak{D}$は空でない弧状連結集合, 
	\item 任意の$D_1, D_2 \in \mathfrak{D}$について, $D_1 \neq D_2$ならば$D_1 \cup D_2 = \emptyset$,
	\item $\bigcup_{D \in \mathfrak{D}}D = \R^n$
\end{itemize}

を満たすとき, $\mathfrak{D}$を$\R^n$の分割という.
\end{definition}

\begin{definition}
$F \subset \R[x_1, \dots, x_n]$を有限部分集合とする.

$D \subset \R^n$が$F$-符号不変であるとは, 任意の$f \in F$に対し, $f$の符号が$D$上一定であることと定義する.

さらに. $\R^n$の分割$\mathfrak{D}$が, 任意の$D \in \mathfrak{D}$に対して, $D$が$F$-符号不変となるとき, $\mathfrak{D}$を$\R^n$の$F$-符号不変な分割という.
\end{definition}

\subsection{描画可能}
\begin{definition}
$F \subset \R^n[x_1, \dots, x_n]$を有限部分集合とする. 空でない弧状連結部分集合$S \subset \R^{n-1}$が$F$-描画可能であるとは, 
\begin{itemize}
	\item 任意の $ x \in S $ に対し, $ F $の解の個数, すなわち $ \{y \in \mathbb{R} \mid \text{ある $f \in F$ に対し f(x,y) = 0} \} $ の元の個数が一定であり, 
	\item 各 $ x \in S $ の $ F $ の解を $ f_1(x) < f_2(x) < \dots < f_k(x) $ と書くとき, 各 $ f_i $ は $ S $ 上の実数値連続関数
\end{itemize}
であることと定義する.
\end{definition}

\begin{proposition}
$S \subset \R^{n-1}$を空でない弧状連結部分集合とし, $F \subset \R[x_1, \dots, x_n]$を有限部分集合とする.
次の3条件を満たすとき, $S$は$F$-描画可能である.
\begin{itemize}
\item 任意の$f \in F$に対し, $S$上$f$の複素数根の数は重複度込みで一定である.
\item 任意の$f \in F$に対し, $S$上$f$の相異なる複素数根の数は一定である.
\item 相異なる任意の$f, g \in F$に対し, $S$上$f, g$に共通する複素数根の数は重複度込みで一定である.
\end{itemize}
\end{proposition}

この命題を示すために次の二つの補題を用意する.

\begin{lemma}
$S \subset \R^{n-1}$を空でない弧状連結部分集合とし, $f_1, f_2 \in \R[x_1, \dots, x_n]$が次を満たすとする.
\begin{itemize}
\item 各$i=1,2$に対し, $S$上$f_i$の重複度込みの複素数根の数は一定.
\item 各$i=1,2$に対し, $S$上$f_i$の相異なる複素数根の数は一定.
\item $S$上$f_1, f_2$の重複度込みの複素数共通根の数は一定.
\end{itemize}
このとき, $S$上$f_1, f_2$の相異なる複素数共通根の数は一定.
\end{lemma}
\begin{proof}
方針: $S_k = \{a \in S \mid f_1(a), f_2(a)の相異なる複素数共通根がk個 \}$が開集合であることを示す. (多項式の解の, 係数についての連続性から示せる.)

\end{proof}

\begin{lemma}
$A \in \R[x_1, \dots, x_n]$とし, $S \subset \R^{n-1}$を弧状連結部分集合とする.
\begin{itemize}
\item $S$上$A$の重複度込みの複素数根の数は一定.
\item $S$上$A$の相異なる複素数根の数は一定.
\end{itemize}
この時, $S$は$\{A\}$-描画可能である.
\end{lemma}

\begin{proof}
方針:次の二つのことを示さなければならない.\\
1.  $S$上実根の数は一定.\\
2.  $S$上実根は連続である.\\
いずれも多項式の根の係数に対する連続性から示せる.
\end{proof}

\begin{proof}[{\bf 命題$1.1.$の証明}]

補題1.2. より, 各$f \in F$に対して, $S$は$\{f\}$-描画可能である.
よって,$S$上の連続関数 $\alpha_{1,f}(a) < \dots \alpha_{n_f, f}(a)$を, 各$a \in S$で$f(a)(x) \in \R[x]$の解であるようにとれる.

\begin{claim*}
$f, g \in F$が, $f \neq g$であるとする,
ある$a \in S$において$\alpha_{k,f}(a) = \alpha_{l,g}(a)$であるならば, 任意の$a \in S$に対して$\alpha_{k,f}(a) = \alpha_{l,g}(a)$である.
\end{claim*}

この主張は, $S$が弧状連結であることと, 補題1.1. から従う. 
この主張より, $F$の解の個数は$S$上一定である. よって, 命題が示された.
\end{proof}

\begin{corollary}
$S \subset \R^{n-1}$を弧状連結部分集合とし, $F \subset \R[x_1,\dots, x_n]$を有限部分集合とする.
次が成り立つとき, $S$は$F$-描画可能である.
\begin{itemize}
\item 任意の$f \in F$に対し, $\deg(f(a))$が一定($a \in S$).
\item 任意の$f \in F$に対し, $\deg(\gcd(f(a), \frac{\partial f}{\partial x_n}(a)))$が一定($a \in S$).
\item 任意の$f, g \in F$に対し, $\deg(\gcd(f(a), g(a)))$が一定($a \in S$).
\end{itemize}
\end{corollary}

\subsection{主部分終結式係数(Principal Subresultant Coefficient)}

ここで書くこと. $\gcd(f(a),g(a))$の次数が$\mathrm{psc}(f(a),g(a))$から決まるということ.


\subsection{符号不変な分割の存在とCADアルゴリズム}
符号不変な分割の存在を示し,CADアルゴリズムについて明記する. 
\end{document}