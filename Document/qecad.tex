\documentclass[uplatex, dvipdfmx]{jsarticle}

\usepackage{amsmath,amssymb}
\usepackage{amsthm}
\usepackage{algorithmic}
\usepackage{algorithm}
\usepackage{cleveref}
\usepackage{todonotes}

\numberwithin{equation}{section}
\makeatletter
\renewenvironment{proof}[1][\proofname]{\par
  \pushQED{\qed}%
  \normalfont \topsep6\p@\@plus6\p@\relax
  \trivlist
  \item\relax
  {\bfseries
  #1\@addpunct{.}}\hspace\labelsep\ignorespaces
}{
  \popQED\endtrivlist\@endpefalse
}
\makeatother

\newcommand{\R}{\mathbb{R}}
\newcommand{\Q}{\mathbb{Q}}
\newcommand{\C}{\mathbb{C}}
\newcommand{\Ralg}{\mathbb{R}_\mathrm{alg}}
\newcommand{\Z}{\mathbb{Z}}
\newcommand{\Qua}{\mathfrak{Q}}
\newcommand{\M}{\mathfrak{M}}
\newcommand{\calL}{\mathcal{L}}
\newcommand{\calS}{\mathcal{S}}
\newcommand{\calP}{\mathcal{P}}
\newcommand{\defiff}{ :\Leftrightarrow}
\newcommand{\RCF}{\mathrm{RCF}}
\newcommand{\Var}{\mathrm{Var}}
\newcommand{\psc}{\mathrm{psc}}
\newcommand{\PROJ}{\mathrm{PROJ}}
\newcommand{\APROJ}{\mathrm{APROJ}}
\newcommand{\der}{\mathrm{der}}
\newcommand{\sign}{\mathrm{sign}}
\newcommand{\SIGN}{\mathrm{SIGN}}
\newcommand{\map}[3]{{#1}:{#2}\rightarrow{#3}}

\theoremstyle{definition}
\newtheorem{definition}{定義}[section]
\crefname{definition}{定義}{定義}
\newtheorem{proposition}{命題}[section]
\crefname{proposition}{命題}{命題}
\newtheorem{lemma}{補題}[section]
\crefname{lemma}{補題}{補題}
\newtheorem{theorem}{定理}[section]
\crefname{theorem}{定理}{定理}
\newtheorem{corollary}{系}[section]
\crefname{corollary}{系}{系}
\newtheorem*{claim*}{主張}
\newtheorem{remark}{注意}[section]
\newtheorem{example}{例}[section]

\renewcommand{\proofname}{\textbf{証明}}

\begin{document}

\title{実閉体上の柱状代数分解による量化記号消去}
\author{富永 直弥}
\maketitle

% mathscinetからbibtex.
% todonote
% 実数体の公理ではない,実閉体の公理だし.例1.2は例じゃなく定義にすべきでは.
% section6の続きをひとまず書く.
% 【前日に1行でも編集した場合はgithubにpushする.】
% 【平日午前9時にdiscordで定時連絡する.】
% この本を呼んでいました,休んでました,等でもOK.

\section{実閉体}

\todo[inline]{
     どの論文で導入されたかの引用をするべき.論文の定理番号も可能なら引用する.
     順序体・実体の概念がいつ頃導入されたものなのかはよくわからないです.
}
まず,\cite{MR3069467}でArtinとSchreierにより導入された,実閉体の概念についてまとめる.
実閉体は,実数体を拡張して定義される概念であり,実閉体上でも実数体と類似した解析的な性質が成り立つ.

まず始めに,順序体と実体の定義を述べる.

\begin{definition}[順序体と実体の定義]
     体$F$と$F$上の全順序構造$<$の組$(F,<)$が次の性質を満たすとき,$(F,<)$を順序体(ordered field)という.
     \begin{enumerate}
          \item 任意の$x,y,z\in F$に対し,$x \leq y$ならば$x + z \leq y + z$である.
          \item 任意の$x,y \in F$に対し,$x \geq 0$, $y \geq 0$ならば$x \cdot y \geq 0$である.
     \end{enumerate}

     また,体$F$が,任意の有限個の$F$の元$x_1, \dots, x_k \in F$について,
     $-1 \neq \sum_{i=1}^k x_i^2$を満たすとき,実体(real field)であるという.
\end{definition}

順序体と実体について,次の命題が成り立つ.

\todo[inline]{ここは命題であったが,定理に修正した}
\begin{theorem}\todo[inline]{ 証明を付けるか,引用するかどちらかすべき.}
     $F$を体とする.このとき,以下は同値である.
     \begin{enumerate}
          \item $F$が実体である.
          \item $F$が順序体となるような全順序構造$<$を持つ.
     \end{enumerate}
\end{theorem}


\todo[inline]{$\Q(\sqrt{2})$を$\Q\left(\sqrt{2}\right)$に修正.以下同様.}
例えば,実数体$\R$や有理数体$\Q$,有理数体$\Q$に$\sqrt{2}$を添加した体$\Q\left(\sqrt{2}\right)$は,
通常の全順序構造を入れることにより順序体となる.
したがって,$\R, \Q, \Q\left(\sqrt{2}\right)$はいずれも実体である.

\todo[inline]{$\C$が順序体の構造を持たないことは命題として述べてもいいのでは.}
一方で,複素数体$\C$は順序体の構造を持たない.よって,$\C$は実体ではない.

有理数体$\Q$は,実体$\Q\left(\sqrt{2}\right)$を代数拡大にもつが,
実数体$\R$は代数拡大で実体になるものは自分以外には持たない.
そのような特徴に注目し,実閉体の概念を導入する.

\todo[inline]{引用の書き方が分からないのですが,これでいいでしょうか?}
実閉体は\cite{MR3069467}により初めて定義された.
なお,\cite[Definition 1.2.1]{MR1659509}の定義を参照している.

\begin{definition}[実閉体の定義] \label{definition:RCF}
     実体$F$が実閉体(real closed field)であるとは,$F$の代数拡大で実体になるものは$F$自身に限ることである.
\end{definition}

先ほど述べたように,有理数体$\Q$は実閉体ではないが,実数体$\R$は実閉体である.
また,代数的数全体(すなわち, 有理数体$\Q$の代数的閉包)$\overline{\Q}$と実数体$\R$の共通部分$\Ralg := \overline{\Q} \cap \R$は実閉体である.
$\Ralg$に属する元を実代数的数という.

\todo[inline]{一階述語論理で書けない~というところを,消去しました.}
この実閉体の\cref{definition:RCF}は,次の定理を用いることで,一階述語論理で実閉体の理論を定式化することができる.

\todo[inline]{
     以下の定理の引用を加えました.(これでよい?)
     1⇒2, 3は示されているようだが,ぱっと見,逆が示されていなそう.
     以下の定理は証明を書いていない.
}

次の定理は,\cite{MR3069467}において示された.
また,\cite[Theorem.1.2.2.]{MR1659509}を参照した.
\begin{theorem}\todo[inline]{定理の証明はされていないので引用をすべき.}
     体$F$について,以下は同値である.
     \begin{enumerate}
          \item 体$F$は実閉体である.
          \item 次を満たす$F$の順序体の構造$<$がただ一つ存在する.
               \begin{itemize}
                    \item 任意の$x \in F$に対し, $x>0$ならば$x = y^2$を満たす$y \in F$が存在する.
                    \item 奇数次数の$F$係数$1$変数多項式は,$F$において少なくとも$1$つの根を持つ.
               \end{itemize}
          \item $F\left[\sqrt{-1}\right] := F[X]/(X^2 + 1)$は代数的閉体である.
     \end{enumerate}
\end{theorem}

\todo[inline]{以降 → これ以下では に訂正.}
これ以下では,実閉体$R$上の区間を,
\begin{equation}
     [a,b]:=\{x \in R \mid a \leq x \leq b\}, \quad (a,b):=\{x \in R \mid a < x < b\}
\end{equation}
のようにかくことにする.

実閉体$R$を係数とする一変数多項式について,実数$\R$上の連続関数,可微分関数と同様の性質が成り立つ.

\begin{proposition}\label{proposition:intermediate}
     $R$を実閉体とし,$f \in R[X]$とする.
     また,$a<b$を満たす$a,b \in R$をとる.
     $f(a)f(b)<0$であれば,ある$x \in (a,b)$が存在し,$f(x)=0$を満たす.
\end{proposition}
\begin{proof}
     $R[\sqrt{-1}]$が代数的閉体であるから,
     $f(X)$の既約因子は,
     \begin{align}
          &X-\alpha, \\
          &(X-\beta)^2 + \gamma^2 = (X-\beta-\gamma\sqrt{-1})(X-\beta+\gamma\sqrt{-1})
     \end{align}
     の形のどちらかである($\alpha, \beta, \gamma \in R$, $\gamma \neq 0$).

     \todo[inline]{証明にギャップがあったので直した.}
     このとき,$(X-\beta)^2 + \gamma^2$の形の既約因子は,$\gamma \neq 0$であるから$R$上で常に$0$より大きいことに注意する.
     $f(a)f(b)<0$であることから,$f(X)$のある既約因子$g(X)$について,$g(a)g(b)<0$となる.
     実際,もしすべての$f(X)$の既約因子$g(X)$について,$g(a)g(b) \geq 0$であれば,$f(a)f(b)\geq0$となり仮定に反する.
     ここで,先に述べた注意から,$g(a)g(b)<0$となるような$f(X)$の既約因子$g(X)$は,$g(X) = X - \alpha$とかける.
     $g(a)g(b) < 0$であることから,$a<\alpha<b$であり,このとき$f(\alpha)=0$である.
     したがって,命題が示された.
\end{proof}

\begin{proposition}\label{proposition:Rolle}
     $R$を実閉体とし,$f \in R[X]$とする.
     また,$a, b \in R$を$a<b$とする.
     $f(a)=f(b)=0$であるとき,ある$x \in (a,b)$が存在し,$f'(x)=0$である.
\end{proposition}
\begin{proof}
     \todo[inline]{以下の主張において,「十分である」ことの説明を加えた.}
     $(a,b)$上に$f(X)$の根がない場合について示せば十分である.
     実際,もし$(a,b)$上に$f(X)$の根が存在するとき,それらの根を$a < x_1 < \dots < x_k < b$とすれば,
     $x_1$を改めて$b$とすることで$(a,b)$上に$f(X)$の根がない場合に帰着する.

     よって以下では,$(a,b)$上に$f(X)$の根がないとする.
     \begin{equation}
     f(X) = (X-a)^m(X-b)^ng(X), \, g(a)\neq0,\, g(b)\neq0
     \end{equation}
     とする.このとき,$(a,b)$上$f(X)$の根がないことから,\cref{proposition:intermediate}より,$g(a)g(b)>0$である.
     
     $g_1(X) := m(X-b)g(X)+n(X-a)g(X)+(X-a)(X-b)g'(X)$とおくと.
     \begin{equation}
          f'(X) = (X-a)^{m-1}(X-b)^{n-1}g_1(X)
     \end{equation}
     である.
     このとき,$g_1(a) = m(a-b)g(a)$, $g_1(b)=n(b-a)g(b)$であるから,$g_1(a)g_1(b)<0$である.
     よって,\cref{proposition:intermediate}より,ある$x \in (a, b)$が存在し$g_1(x)=0$である.
     この$x \in (a,b)$に対して$f'(x)=0$である.
\end{proof}

\todo[inline]{この系と証明を追加した.}
\begin{corollary}\label{corollary:mean-value}
     $R$を実閉体とし,$f \in R[X]$とする.また,$a, b \in R$を$a<b$とする.
     このとき,ある$c \in (a,b)$が存在して,$f(b)-f(a) = (b-a)f'(c)$を満たす.
\end{corollary}

\begin{proof}
     $F(X) \in R[X]$を,
     \begin{equation}
          F(X) := f(X) - \left\{\frac{f(b)-f(a)}{b-a}(X-a) + f(a)\right\}
     \end{equation}
     と定義すると,$F(a)=F(b)=0$であるから,\cref{proposition:Rolle}より,
     ある$c \in (a,b)$が存在して,$F'(c)=0$を満たす.
     この$c$に対して,$f(b)-f(a)=f'(c)(b-a)$が成り立つ.
\end{proof}

\todo[inline]{この系の証明を追加した.}
\begin{corollary}\label{corollary:monotone}
     $R$を実閉体とし,$f \in R[X]$とする.また,$a, b \in R$を$a<b$とする.
     このとき,任意の$x \in (a,b)$について$f'(x)>0$ならば,$f(X)$は$[a,b]$上狭義単調増加である.
     また,任意の$x \in (a,b)$について$f'(x)<0$ならば,$f(X)$は$[a,b]$上狭義単調減少である.
\end{corollary}
\begin{proof}
     後半も同様に示せるので,前半の主張を示す.
     $f(X)$が$[a,b]$上狭義単調増加でないとすると,ある$x \in (a,b)$について$f'(x) \leq 0$となることを示す.
     $f(X)$が$[a,b]$上狭義単調増加でないとすると,
     $a \leq x_0 < x_1 \leq b$である$x_0, x_1 \in R$が存在して,$f(x_0) \geq f(x_1)$を満たす.
     このとき,\cref{corollary:mean-value}より,$x_0 < x < x_1$である$x \in (a,b)$が存在し,
     $f(x_1) - f(x_0) = f'(x)(x_1 - x_0)$を満たす.
     この$x \in (a,b)$に対して$f'(x) \leq 0$となり,前半の主張の証明が完了した.
\end{proof}

\section{一階述語論理}
一階述語論理式(first-order logic)は,
\todo[inline]{変数記号,論理記号,量化記号を英語でも何というかを書くべき.}
変数記号,論理記号 ($=, \lnot, \lor, \land, \rightarrow$), 量化記号($\forall, \exists$)
と,言語$L$で記述される.

\todo[inline]{
     言語も何と書くか書くべき,
     述語記号,定数記号,関数記号の集合を区別するべき.
     このあたりはよく検討するべき.教科書に合わせたほうが良さそう.
}
ここで,言語$L$とは,述語記号,関数記号,定数記号からなる集合である.
ただし,各述語記号$R \in L$は,何個の変数を受け取るか定まっているものとし,
同様に各関数記号$f \in L$は,何個の変数を受け取るか定まっているものとする.
$n$個の変数を受け取る述語記号を$n$変数述語記号といい,$n$個の変数を受け取る関数記号を$n$変数関数記号という.

\begin{example}[順序環の言語]
     次の記号からなる言語$L_\mathrm{OR}$を,順序環の言語という.
     \begin{itemize}
          \item 2変数述語記号 $<$,
          \item 2変数関数記号 $+$, $\cdot$,
          \item 定数記号 $1$, $0$.
     \end{itemize}
\end{example}

言語$L$が定義されると,順に項,原子論理式,論理式が定義される.

\begin{definition}[項の定義]
     言語$L$に対して,項(term)が帰納的に定義される.
     \begin{enumerate}
          \item 変数記号,定数記号は項である.
          \item $t_1, \dots, t_n$がいずれも項とする.任意の$n$変数関数記号$f \in L$に対し,$f(t_1, \dots, t_n)$は項である.
          \item 以上の規則により定まる記号列のみが項である.
     \end{enumerate}
     言語$L$上の項であることを強調したいとき,$L$項($L$-term)と呼ぶ.
\end{definition}

\todo[inline]{$R$は$n$変数述語記号とする,のように書くべき.$R \in L$のような書き方を避けるべき.}

\begin{definition}[論理式の定義]
     言語$L$に対して,原子論理式(atomic formula)を,次のいずれかの形をした記号列で定義する.
     \begin{enumerate}
          \item $t_1, t_2$を項として,$(t_1=t_2)$.
          \item $R \in L$を$n$変数述語記号,$t_1, \dots, t_n$を項として,$(R(t_1, \dots, t_n))$.
     \end{enumerate}
     
     また,原子論理式$\phi$またはその否定$(\lnot \phi)$を,リテラル(literal)と呼ぶ.
     さらに,言語$L$に対して,論理式(formula)が帰納的に定義される.
     \begin{enumerate}
          \item 原子論理式は論理式である.
          \item $\phi$が論理式であるとすると,$(\lnot \phi)$は論理式である.
          \item $\phi$,$\psi$が論理式であるとすると,$(\phi \lor \psi)$, $(\phi \land \psi)$, $(\phi \rightarrow \psi)$はいずれも論理式である.
          \item $\phi$が論理式であるとすると,任意の変数記号$x$に対して$(\forall x \phi)$, $(\exists x \phi)$は論理式である.
          \item 以上の規則により定まる記号列のみが論理式である.
     \end{enumerate}
     言語$L$の論理式であることを強調したいときは,$L$論理式($L$-formula)と呼ぶ.
\end{definition}

論理式$\phi := \forall x (x^2 + y > 0)$について観察する.ここで,$x,y$はいずれも変数記号である.

これは,「全ての$x$に対して$x^2 + y > 0$が成り立つ」というように読むことができるが,$y$の値は自由に設定することができ,$y$の値が定まらない限りは真であるか偽であるかははっきりしない.
この$y$のように,自由な状態にある変数記号を論理式$\phi$の自由変数といい,一方$x$のように自由な状態にない変数記号を論理式$\phi$の束縛変数という.

厳密には,次のように定義する.

\begin{definition}[自由変数の定義]
     言語$L$を一つとる.
     項$t$に登場する全ての変数記号を$\Var(t)$で表す.

     論理式$\phi$の自由変数(free variable)全体$\Var(\phi)$を,次のように論理記号及び量化記号の数に関して帰納的に定義する.
     \begin{enumerate}
          \item $t_1, t_2$を項とするとき,$\Var((t_1=t_2)):=\Var(t_1)\cup\Var(t_2)$と定義する.
          \item $R \in L$を$n$変数述語記号,$t_1, \dots, t_n$を項とするとき,$\Var(R(t_1, \dots, t_n)):= \bigcup_{i=1}^n \Var(t_i)$と定義する.
          \item 論理式$\phi$に対して$\Var(\phi)$が定義されているとき,$\Var((\lnot \phi)):=\Var(\phi)$と定義する.
          \item 論理式$\phi, \psi$に対して$\Var(\phi), \Var(\psi)$が定義されているとき,
          $\Var((\phi \lor \psi)):=\Var(\phi)\cup\Var(\psi)$,
          $\Var((\phi \land \psi)):=\Var(\phi)\cup\Var(\psi)$,
          $\Var((\phi \rightarrow \psi)):=\Var(\phi)\cup\Var(\psi)$
          と定義する.
          \item 論理式$\phi$に対して,$\Var(\phi)$が定義されているとき,変数記号$x$に対して
          $\Var((\forall x\phi)):=\Var(\phi) \setminus \{x\}$,
          $\Var((\exists x\phi)):=\Var(\phi) \setminus \{x\}$
          と定義する.
     \end{enumerate}

     論理式$\phi$に登場する自由変数が高々$x_1, \dots, x_n$である事を明示したいとき,$\phi(x_1, \dots, x_n)$と書く.ただし,登場しない自由変数があっても良い.

     項$t$が自由変数を持たないとき,すなわち$\Var(t)=\emptyset$であるとき,$t$を閉項(closed term)という.
     また,論理式$\phi$が自由変数を持たないとき,すなわち$\Var(\phi)=\emptyset$であるとき,$\phi$を閉論理式(closed formula)という.
\end{definition}

\begin{definition}[理論]
     言語$L$に対して,閉論理式からなる集合を$L$理論($L$-theory),あるいは単純に理論(theory)という.
\end{definition}


\todo[inline]{
     ここは信頼できる文献を参照して書いたほうがいい.   
     最後の主張は正しくない.
     AxiomScheme(公理図式)なので,そもそも公理じゃない.
}

\cite{Arai}に従い,実閉体の理論$RCF$を定義する.

\begin{example}[実閉体の理論]
     順序環の言語$L_\mathrm{OR}=\{<, +, \cdot, 1, 0\}$上の実閉体の理論$\RCF$を,次のように定義する.
     
     \begin{itemize}
          \item 体の公理
          \begin{enumerate}
               \item $\forall x \forall y(x + y = y + x)$
               \item $\forall x \forall y \forall z((x + y) + z = x + (y + z))$
               \item $\forall x (x + 0 = x)$
               \item $\forall x \exists y (x + y = 0)$
               \item $\forall x \forall y(x \cdot y = y \cdot x)$
               \item $\forall x \forall y \forall z((x \cdot y) \cdot z = x \cdot (y \cdot z))$
               \item $\forall x (x \cdot 1 = x)$
               \item $\forall x \exists y ((x=0) \lor (x \cdot y = 1))$
               \item $\lnot(0 = 1)$
          \end{enumerate}
          \item 全順序の公理
          \begin{enumerate}
               \item $\forall x (\lnot(x<x))$
               \item $\forall x \forall y \forall z((x < y \land y < z) \rightarrow x < z)$
               \item $\forall x \forall y (\lnot(x = y) \rightarrow (x<y \lor y<x))$
          \end{enumerate}
          \item 順序体の公理
          \begin{enumerate}
               \item $\forall x \forall y \forall z (x \leq y \rightarrow x + z \leq y + z)$
               \item $\forall x \forall y ((0 \leq x \land 0 \leq y) \rightarrow 0 \leq x \cdot y )$
          \end{enumerate}
          \item 実数体の公理
          \begin{enumerate}
               \item $\forall x \exists y (x=y^2 \lor -x=y^2)$
               \item $\forall a_{2n} \dots \forall a_1 \forall a_0 \exists x(x^{2n+1} + a_{2n}\cdot x^{2n} + \dots +  a_1 \cdot x + a_0 = 0)) \quad (n=0,1,2,\dots)$ 
          \end{enumerate}
     \end{itemize}

     ただし,項$t_1, t_2$に対し,$t_1 \leq t_2$は$(t_1 < t_2)\lor(t_1 = t_2)$の省略記号とする.
     また,項$t$と$n=1,2,\dots$に対し,$t^n$は,$t^n:=t \cdot t^{n-1}$, $t^1:=t$により帰納的に定義された省略記号とする.
\end{example}

理論$\RCF$の閉論理式をすべて満たすような「数学的構造」を,実閉体とよびたい.
このような定義を正当化させるには,この「数学的構造」が,言語$L_\mathrm{OR}$で書かれた閉論理式を,真か偽か判定できることが必要である.
そのために,言語を解釈することのできる「数学的構造」を定義する.

\begin{definition}[構造の定義]
     言語$L$において,$L$構造($L$-structure)とは,空でない集合$M$と,次のようなデータの組$\M=(M,\dots)$である.
     \begin{itemize}
          \item 各$n$変数述語記号$R \in L$に対応する部分集合$R^\M \subset M^n$.
          \item 各$n$変数関数記号$f \in L$に対応する写像$\map{f^\M}{M^n}{M}$.
          \item 各定数記号$c \in L$に対応する元$c^\M \in M$.
     \end{itemize}
\end{definition}

$L$構造$\M$が与えられたとき,
$L$論理式$\phi$が$L$構造$\M$で成り立つかどうか,すなわち充足関係があるかどうかを定義できる.

\todo[inline]{定義は引用した方がいい.}
\begin{definition}
     $L$-構造$\M = (M,\dots)$をとる.
     言語$L$に形式的な定数記号$\{c_m \mid m \in M\}$を追加した言語を$L(\M)$とする.
     ここで,各$m \in M$に対し, $c_m^{\M}:=m$とすることで,
     $\M$は$L(\M)$構造に拡張できる.
     \begin{itemize}
          \item $L(\M)$閉項$t$の解釈$t^\M$を帰納的に定義する.
          \begin{enumerate}
               \item $f$を$n$変数関数記号とし, 閉項$t_1$, \dots $t_n$のそれぞれの解釈$t_1^\M,\dots,t_n^\M$が定まっているとする.
               このとき,$(f(t_1,\dots,t_n))^\M:=f^\M(t_1^\M,\dots,t_m^\M)$と定める.
          \end{enumerate}
          \item $L(\M)$閉論理式$\phi$に対し, 充足関係(satisfaction relation)があること($\M \models \phi$とかく)を帰納的に定義する.
          \begin{enumerate}
               \item $t_1,t_2$を$L(\M)$上の閉項とするとき, 
               \begin{equation}
                    \M \models (t_1 = t_2) \defiff t_1^\M = t_2^\M,
               \end{equation}
               \item $t_1,\dots t_n$を$L(\M)$閉項,$R$を$n$変数述語記号とするとき,
               \begin{equation}
                    \M \models (R(t_1, \dots, t_n)) \defiff (t_1^\M, \dots, t_n^\M) \in R^\M,
               \end{equation}
               \todo[inline]{$\not \models$ は定義されていない.}
               \item $L(\M)$閉論理式$\phi$に対し
               \begin{equation}
                    \M \models (\lnot \phi) \defiff \M \not\models \phi,
               \end{equation}
               \item $L(\M)$閉論理式$\phi, \psi$に対し
               \begin{align}
                    &\M \models (\phi \lor \psi) \defiff \M \models \phi \text{または} \M \models \psi,\\
                    &\M \models (\phi \land \psi) \defiff \M \models \phi \text{かつ} \M \models \psi,\\
                    &\M \models (\phi \rightarrow \psi) \defiff \M \not\models \phi \text{または} \M \models \psi,
               \end{align}
               \item 変数記号が高々$x$のみである$L(\M)$論理式$\phi(x)$に対し
               \begin{align}
                    &\M \models (\forall x \phi(x)) \defiff \text{任意の$m \in M$に対し}\M \models \phi(c_m),\\
                    &\M \models (\exists x \phi(x)) \defiff \text{ある$m \in M$に対し}\M \models\phi(c_m).
               \end{align}
          \end{enumerate}
     \end{itemize}
\end{definition}

\begin{definition}
     $L$を言語とし,$T$を$L$理論とする.
     $L$構造$\M$が任意の$\phi \in T$に対して$\M\models \phi$であるとき,$L$構造$\M$を$T$のモデル(model)といい,$\M \models T$とかく.
\end{definition}

順序環の言語$L_{\mathrm{OR}}$において,$L_\mathrm{OR}$構造$\R=(\R,<,+,\cdot,1,0)$や$\Ralg=(\Ralg,<,+,\cdot,1,0)$は,
\begin{equation}
     \R \models \RCF, \quad \Ralg \models \RCF
\end{equation}
であり,構造$\R$,$\Ralg$はいずれも実閉体の理論$\RCF$のモデルである.

\section{量化記号消去}

\begin{example}
     言語$L_\mathrm{OR}$上の次のような論理式を考える.
     \begin{equation}
          \phi(a) := \forall x (x^2 + ax + a + 3 > 0)
     \end{equation}

     簡単のため,実閉体のモデルである実数体$\R$をとって考える.すなわち,上記の論理式は次のように言い換えられる.
     \begin{equation}
          \text{全ての実数$x \in \R$に対して$x^2 + ax + a + 3 > 0$をみたす.}  
     \end{equation}
     これを,変数記号$a$についての条件と解釈すると,2次方程式$x^2 + ax + a + 3 = 0$の判別式が0より大きいことに必要十分であるから,
     \begin{equation}
          \psi(a) := (a < -2) \lor (6 < a)
     \end{equation}
     に同値である.すなわち,次が成り立つ.
     \begin{equation}
          \R \models \forall a (\phi(a) \leftrightarrow \psi(a)).
     \end{equation}
\end{example}

上記の例は,実数体$\R$についてのみ考察しているが,実は,全ての$\RCF$のモデル$R$に対して
\begin{equation}
     R \models \forall a (\phi(a) \leftrightarrow \psi(a))
\end{equation}
が成り立つ.ここで,$\phi$には束縛変数$x$が現れており,一方で$\psi$に登場する変数記号は全て自由変数である.
つまり,$\phi$に登場する量化記号を消去し,自由変数のみの論理式$\psi$で記述できている.このような現象を扱うために,「量化記号を消去する」概念を定義する.


\begin{definition}
$T$を言語$L$上の理論とし,$\phi$を$L$上の論理式とする.
全ての$T$のモデル$\M$に対して$\M \models \phi$が成り立つとき,$\phi$は$T$の帰結(consequence)であるといい,$T \models \phi$とかく.

\todo[inline]{理論$T$が量化記号消去を「持つ」と書くべきか? 検討する.以降の表現も気を付ける.}
理論$T$で量化記号消去(quantifier elimination)ができるとは,
$L$上の任意の論理式$\phi(x_1,\dots,x_n)$に対し,量化記号のない論理式$\psi(x_1,\dots, x_n)$が存在し,
\begin{equation}
     T \models \forall x_1 \dots \forall x_n(\phi(x_1,\dots,x_n) \leftrightarrow \psi(x_1, \dots, x_n))
\end{equation}
が成り立つことと定義する.
\end{definition}


実閉体の理論は量化記号消去ができる.すなわち,次が成り立つ.

\todo[inline]{これは論文を引用するべき.discordに挙げてくれた.}
\begin{theorem}[Tarski]\label{theorem:Tarski}
     実閉体の理論$\RCF$は量化記号消去ができる.
\end{theorem}

\todo[inline]{「以降この節は」「この節の残りでは」等に直すべき.でもそれも砕けすぎている.}
以降この節は\cref{theorem:Tarski}の証明を述べる.

この定理の証明および後々の議論のため,言語$L_\mathrm{OR} = \{<, +, \cdot, 1, 0\}$上の原子論理式について注意をしておく.

まず,正整数の記号$n$を,$1$の$n$個の和$(1 + (1 + \dots + 1 ))$の省略記号であるとする.

さらに,$x_1, \dots, x_n$を変数記号として,多項式$f \in \Z[x_1, \dots, x_n]$が,$a_1, \dots, a_k, b_1, \dots, b_l$を正整数として,
\begin{equation}
     f(x_1, \dots, x_n) = \sum_{i=1}^k a_i x_1^{\alpha_{i,1}} \cdots x_n^{\alpha_{i,n}} - \sum_{j=1}^l b_j x_1^{\beta_{j,1}} \cdots x_n^{\beta_{j,n}}
\end{equation}
で与えられているとする.

このとき,以下のように原子論理式の省略記号を導入する.
\begin{align}
     \sign (f(x_1, \dots, x_n)) = 1 &\defiff \sum_{i=1}^k a_i x_1^{\alpha_{i,1}} \cdots x_n^{\alpha_{i,n}} > \sum_{j=1}^l b_j x_1^{\beta_{j,1}} \cdots x_n^{\beta_{j,n}}\\
     \sign (f(x_1, \dots, x_n)) = 0 &\defiff \sum_{i=1}^k a_i x_1^{\alpha_{i,1}} \cdots x_n^{\alpha_{i,n}} = \sum_{j=1}^l b_j x_1^{\beta_{j,1}} \cdots x_n^{\beta_{j,n}}\\
     \sign (f(x_1, \dots, x_n)) = -1 &\defiff \sum_{i=1}^k a_i x_1^{\alpha_{i,1}} \cdots x_n^{\alpha_{i,n}} < \sum_{j=1}^l b_j x_1^{\beta_{j,1}} \cdots x_n^{\beta_{j,n}}
\end{align}

\todo[inline]{
     を用いて→に対して\\
     ~しておくは砕けた表現.注意する,とする,というようにする.
}
このとき,$\RCF$において,任意の原子論理式$\phi(x_1, \dots, x_n)$は,ある多項式$f \in \Z[x_1, \dots, x_n]$及び$\sigma \in \{-1, 0, 1\}$を用いて,
$\sign(f(x_1, \dots, x_n)) = \sigma$と同値であることに注意しておく.
また,$P = \{-1, 0, 1\}$としておく.

それでは,\cref{theorem:Tarski}を示していく.そのために,まずは次の補題を準備する.

\todo[inline]{補題番号をローマ数字にしてフォントをかえる,アルファベットに直すとかが適切です.「1.より」とかはだめ.}
\begin{lemma}\label{lemma:qe_simplify}
     言語$L_\mathrm{OR}$上の理論$\RCF$において,次の主張が成り立てば.理論$\RCF$は量化記号消去できる.

     \begin{enumerate}
          \item \label{qe_1}
          任意の$f_1, \dots, f_s \in \Z[X, Y_1, \dots, Y_n]$および$\map{\sigma}{\{1,\dots, s\}}{P}$に対して,量化記号のない論理式$\theta(Y_1, \dots, Y_n)$が存在し,
          \begin{equation}
               \RCF \models \forall Y_1 \dots \forall Y_n ( \theta(Y_1, \dots, Y_n) \leftrightarrow \exists X(\bigwedge_i (\sign(f_i(X,Y_1, \dots, Y_n))= \sigma(i))) )
          \end{equation}
          を満たす.
     \end{enumerate}
\end{lemma}

\begin{proof}
     まず,\ref{qe_1}により次の主張が成り立つことを示す.
     \begin{enumerate}
          \setcounter{enumi}{1}
          \item \label{qe_2}
          任意の量化記号のない論理式$\phi(X, Y_1, \dots, Y_n)$に対し,ある量化記号のない論理式$\theta(Y_1, \dots, Y_n)$が存在し,
          \begin{equation}
               \RCF \models \forall Y_1 \dots \forall Y_n(\theta(Y_1, \dots, Y_n) \leftrightarrow \exists X \phi(X, Y_1, \dots, Y_n))
          \end{equation}
          をみたす.
     \end{enumerate}
     実際,任意に与えられた量化記号のない論理式$\phi(X, Y_1, \dots, Y_n)$は,多項式$f_{i,j} \in \Z[X, Y_1, \dots, Y_n]$($i=1, \dots, k, j=1, \dots, l_i$)
     と$\map{\sigma_i}{\{1, \dots, l_i\}}{P}$を用いて,和積標準形\todo[inline]{ 「を用いて,表す」は違和感ない.和関標準形は定義していない.}
     \begin{equation}
          \phi(X, Y_1, \dots, Y_n) \leftrightarrow \bigvee_i \bigwedge_j (\sign(f_{i,j}(X, Y_1, \dots, Y_n)) = \sigma_i(j))
     \end{equation}
     で表すことができる.このとき,
     \begin{equation}
          \exists X \phi(X, Y_1, \dots, Y_n) \leftrightarrow \bigvee_i (\exists X \bigwedge_j (\sign(f_{i,j}(X, Y_1, \dots, Y_n)) = \sigma_i(j)))
     \end{equation}
     が成り立つ.ここで,\ref{qe_1}より,各$i=1, \dots, k$に対して,
     \begin{equation}
          \exists X \bigwedge_j (\sign(f_{i,j}(X, Y_1, \dots, Y_n)) = \sigma_i(j)) \leftrightarrow \theta_i(Y_1, \dots, Y_n)
     \end{equation}
     を満たす量化記号のない論理式$\theta_i(Y_1, \dots, Y_n)$が存在する.
     \todo[inline]{このしたがってはギャップがある.}
     したがって,\ref{qe_2}は成り立つ.

     次に,\ref{qe_2}より,理論$\RCF$が量化記号消去できることを示す.
     任意の論理式$\phi$に対し,それと同値となるような量化記号のない論理式$\theta$が存在することを示せばよい.
     これは,長さに関する帰納法で示す.まず,論理式$\phi$の量化記号が0個,すなわち原子論理式の場合は,すでに量化記号消去できているのでよい.
     次に,論理記号が$n$個以下の場合に成り立っていると仮定して,論理式$\phi$の量化記号が$(n+1)$個の場合に示す.

     \begin{itemize}
          \item $\phi = \lnot \psi$の場合,$\psi$と同値な量化記号のない論理式$\theta$が存在するので,$\phi$は$\lnot \theta$と同値である.
          \item $\phi$が$\psi_1 \land \psi_2$, $\psi_1 \lor \psi_2$, $\psi_1 \rightarrow \psi_2$のいずれかの場合,否定の場合と同様にして示すことができる.
          \item $\phi = \exists X \psi$の場合,帰納法の仮定より,$\psi$は量化記号のない論理式$\theta_1$に書き直すことができる.
          よって,主張\ref{qe_2}より,$\exists X \theta_1$は量化記号のない論理式$\theta_2$に同値である.よって,示された.
          \item $\phi = \forall X \psi$の場合,まずこの論理式は$\lnot \exists X \lnot \psi$に同値であり,
          帰納法の仮定より$\psi$は量化記号のない論理式$\theta_1$に書き直すことができる.
          また,主張\ref{qe_2}より,$\exists X \lnot \theta_1$は量化記号のない論理式$\theta_2$に書き直すことができる.
          このとき,$\phi$は$\lnot \theta_2$に同値となり,示された.
     \end{itemize}     

     以上より,証明が完了した.
\end{proof}

この補題により,\cref{theorem:Tarski}は,次のように言い換えられる.

\begin{theorem}\label{theorem:weak_Tarski}
     任意の$f_1, \dots, f_s \in \Z[X,Y_1, \dots, Y_n]$および$\map{\sigma}{\{1, \dots, s\}}{P}$に対して,量化記号のない論理式$\theta(Y_1, \dots, Y_s)$が存在して,次を満たす.

     任意の実閉体$R$と,任意の$y_1, \dots, y_n \in R$に対し,次の二つが同値になる.
     \begin{enumerate}
          \item $X$についての方程式$\bigwedge_i (\sign_R(f_i(X,y_1, \dots, y_n)) = \sigma(i))$ が,$R$上解を持つ.
          \item 実閉体$R$における解釈$\theta^R(y_1, \dots, y_n)$は真.
     \end{enumerate}
\end{theorem}

ここで,$a \in R$に対して,$\sign_R(a)$は
\begin{equation}
     \sign_R(a) :=
     \begin{cases}
          1 & a > 0 \\
          0 & a = 0 \\
          -1 & a < 0
     \end{cases}
\end{equation}
であるとする.これを,$a$の符号と呼ぶ.

さらに,実閉体$R$と,多項式$f_1, \dots, f_s \in R[X]$に対し,
$\SIGN_R(f_1, \dots, f_s)$を次のように定める.

恒等的に0でないような全ての$f_i$の$R$上の根を$x_1< \dots< x_N$とする.
また,$x_0 = -\infty$, $x_{N+1} = \infty$とする.
このとき,\cref{proposition:intermediate}より,各$f_i$の符号は,$(x_j, x_{j+1})$上一定である.

この時,$s$行$2N+1$列行列$\SIGN_R(f_1, \dots, f_s) = (c_{i,j})_{i=1, \dots, s, j=1, \dots, 2N+1}$を,
\begin{align}
     c_{i,2k} &= \sign_R(f_i(x_k)) \quad k=1, \dots, N\\
     c_{i,2k+1} &= \sign_R(f_i((x_k, x_{k+1}))) \quad k=0, \dots, N
\end{align}
と定める.

このとき,正整数$s, m$に対して,$P_{s,m}$を,$P$成分の$s$行$(2l+1)$列行列($0 \leq l \leq sm$)全体を表すとすると,
$\SIGN_R(f_1, \dots, f_s) \in P_{s,m}$である.

\todo[inline]{fが多変数だったのに,急に1変数になったりして誤解を招いてるかもしれない?
RealAlgebraicGeometry の1章最後にbibliographicnoteがあるのでそこに書いてある参考文献を引用するべきなのでは}
 

\begin{lemma}\label{lemma:qe_1ststep}
     与えられた$\map{\sigma}{\{1, \dots, t\}}{P}$に対して,部分集合$P(\sigma) \subset P_{s,m}$を,次を満たすように取れる.
     任意の実閉体$R$と,次数が$m$以下であるような任意の$f_1, \dots, f_s \in R[X]$に対し,以下が同値である.
     \begin{enumerate}
          \item $\SIGN_R(f_1, \dots, f_s) \in P(\sigma)$
          \item $\bigwedge_i (\sign_R(f_i(X)) = \sigma(i)$)
     \end{enumerate}
\end{lemma}

\begin{proof}
     \begin{equation}
          P(\sigma):=\{A \in P_{s,m} \mid \text{行列$A$のある列が$(\sigma(i))_{i=1}^s$である}\}
     \end{equation}
     とすればよい.
\end{proof}

次の補題は,\cref{theorem:weak_Tarski}の証明の中心となる補題である.

\begin{lemma}\label{lemma:qe_lowering}
     ある$\map{\phi}{P_{2s,m}}{P_{s,m}}$が存在し,次をみたす.

     次数が$m$以下であるような任意の$f_1, \dots, f_s \in R[X]$で,$f_s$は定数でなく,$f_1, \dots, f_{s-1}$は恒等的に$0$でないとする.
     このとき,$g_1, \dots, g_s \in R[X]$を,それぞれ$f_s$を$f_1, \dots, f_{s-1}$, $f_s'$で割ったあまりとすれば,
     \begin{equation}
          \SIGN_R(f_1, \dots, f_s) = \phi(\SIGN_R(f_1, \dots, f_{s-1}, f_s', g_1, \dots, g_s))
     \end{equation}
     である.
\end{lemma}

この補題は,具体的に多項式$f_1, \dots, f_s \in R[X, Y_1, \dots, Y_s]$が与えられた際に,
$w:=\SIGN_R(f_1, \dots, f_{s-1}, f_s', g_1, \dots, g_s)$の行列成分に関する情報のみから
$\SIGN_R(f_1, \dots, f_s)$を構成できることで示せばよい. 

\begin{proof}
     まず,$x_1 < \dots < x_N$を,$f_1, \dots, f_{s-1}, f_s', g_1, \dots, g_s$のうち恒等的に$0$でないものの$R$上の根全体とする.
     この部分列$x_{i_1} < \dots < x_{i_M}$を,$f_1, \dots, f_{s-1}, f_s'$の$R$上の根全体とする.
     この部分列$\{i_k\}_{k=1}^M$は,行列$w$の情報のみから構成できる.

     ここで,$i_0 = 0, i_{M+1} = N+1$としておき,また,$x_0= -\infty, x_{N+1} = \infty$としておく.

     $\map{\theta}{\{1, \dots, M\}}{\{1, \dots, s\}}$を,
     $\theta(k) \in \{1, \dots, s-1\}$かつ$f_{\theta(k)}(x_{i_k}) = 0$,
     または$\theta(k)=s$かつ$f_{s}'(x_{i_k})=0$であるようにとる.
     この$\theta$も,行列$w$の情報のみから構成できる.

     この$\theta$に対して,各$k=1, \dots, M$で$f_s(x_{i_k}) = g_\theta(k)(x_{i_k})$が成り立つ.

     これで,各$x_{i_k}$における$f_s(x_{i_k})$の符号については情報が得られた.
     次に問題になるのは,各$k=1,\dots, k_s$に対し,区間$(x_{i_k},x_{i_{k+1}})$における$f_s(X)$の符号についてである.
     多項式$f_s(X)$は,区間$(x_{i_k},x_{i_{k+1}})$において$f_s'(X)$の符号が$0$でない一定の符号であることから,\cref{corollary:monotone}より,
     この区間の上では狭義単調増加または狭義単調減少である.
     よって,区間$(x_{i_k},x_{i_{k+1}})$上の$f_s(X)$の根の数は高々$1$つである.

     各区間に$f_s(X)$が根を持つことの必要十分条件は,次のように述べれる.
     \begin{itemize}
          \item $M\neq0$で,$k=1, \dots, M-1$に対して,$f_s(X)$が区間$(x_{i_k}, x_{i_{k+1}})$上に解を持つことの必要十分条件は
          \begin{equation}
               \sign_R(g_\theta(k)(x_{i_k}))\sign_R(g_\theta(k+1)(x_{i_{k+1}})) = -1.
          \end{equation}
          \item $M\neq0$で,$(-\infty, x_1)$に解を持つことの必要条件は,
          \begin{equation}
               \sign_R(f_s'((-\infty, x_1))) \sign_R(g_\theta(1)(x_1)) = 1.
          \end{equation}
          \item $M\neq0$で,$(x_M, \infty)$に解を持つことの必要条件は,
          \begin{equation}
               \sign_R(f_s'((x_M, \infty))) \sign_R(g_\theta(M)(x_M)) = -1.
          \end{equation}
          \item $M = 0$の場合は,$(\infty, \infty)$上に必ず解を持つ.
     \end{itemize}
     いずれの条件も,行列$w$の情報のみで記述されることに注意する.

     いま,$y_1 < \dots < y_L$を,$f_1, \dots, f_{s-1}, f_s$の$R$上の根とし,$y_0 = -\infty$, $y_0$ = 0とする.
     $\map{\rho}{\{0, \dots, L+1\}}{\{0, \dots, M+1\} \cup \{(k,k+1) \mid k=0, \dots, M\}}$を,
     \begin{equation}
          \rho(l) := \begin{cases} 
               k & y_l = x_{i_k},\\
               (k, k+1) & y_l \in (x_{i_k}, x_{i_{k+1}})
          \end{cases}
     \end{equation}
     と定める.$\rho$の定義は,行列$w$の情報だけに依存する.

     それでは,$\SIGN_R(f_1, \dots, f_s)$が$w$の情報のみに依存して定まる事を示す.
     
     まず,$i=1, \dots, s-1$に対しては,
     \begin{equation}
          \sign_R(f_i(y_l)) = \begin{cases}
               \sign_R(f_j(x_{i_k})) & \rho(l) = k,\\
               \sign_R(f_j((x_{i_k}, x_{i_{k+1}}))) & \rho(l) = (k, k+1)
          \end{cases}
     \end{equation}
     であり,また,
     \begin{equation}
          \sign_R(f_i((y_l,y_{l+1}))) = \sign_R(f_i((x_{i_k}, x_{i_{k+1}}))) \quad \text{$\rho(l)=k$または$\rho(l)=(k,k+1)$}
     \end{equation}
     である.さらに,$i=s$に対しては,
     \begin{equation}
          \sign_R(f_s(y_l)) = \begin{cases}
               \sign_R(g_{\theta(k)}(x_{i_k})) & \rho(l) = k,\\
               0 & \rho(l) = (k, k+1)
          \end{cases}
     \end{equation}
     であり,また,
     \begin{equation}
          \sign_R(f_s((y_l,y_{l+1}))) = \begin{cases}
               \sign_R(g_{\theta(k)}(x_{i_k})) & \rho(l) = k, l \neq 0, g_{\theta(k)}(x_{i_k})\neq 0\\
               \sign_R(f_s'((x_{i_k},x_{i_{k+1}}))) & \rho(l)=k, l \neq 0, g_{\theta(k)}(x_{i_k}) = 0 \\
               \sign_R(f_s'((x_{i_k}, x_{i_{k+1}}))) & \rho(l) = (k, k+1), l \neq 0\\
               -\sign_R(f_s'((-\infty, x_1))) & l=0
          \end{cases}
     \end{equation}
     である.

     以上により$\SIGN_R(f_1, \dots, f_s)$が$w$の情報飲みに依存して定まることが示せ,よって証明が完了した.
\end{proof}

以上の準備をもとに,\cref{theorem:weak_Tarski}を示す.

\begin{proof}[\cref{theorem:weak_Tarski}の証明]
     \cref{lemma:qe_1ststep}により,次の主張を示せば十分である.
     \begin{claim*}
          任意に$f_1, \dots, f_s \in \Z[X, Y_1, \dots, Y_n]$をとる.
          $f_i$の$X$についての次数の最大値を$m$とする.任意の$W \subset P_{s,m}$に対し,
          ある量化記号のない論理式$\theta(Y_1, \dots, Y_n)$が存在して次を満たす.

          任意の実閉体$R$と,任意の$y_1, \dots, y_n \in R$に対し,以下の二つが同値である.
          \begin{enumerate}
               \item $\SIGN_R(f_1, \dots, f_s) \in W$である.
               \item 実閉体$R$における解釈$\theta^R(y_1, \dots, y_n)$は真.
          \end{enumerate}
     \end{claim*}
     実際,上の主張で$W = P(\sigma)$とすれば,\cref{theorem:weak_Tarski}が従う.

     まず,$m=0$の場合,すなわち$f_1, \dots, f_s \in \Z[Y_1, \dots, Y_n]$の場合,
     $W':=\{A = (a_i)_{i=1}^s \in W \mid \text{Aはs行1列の行列}\}$として
     \begin{equation}
          \theta(Y_1, \dots, Y_n) := \bigvee_{A \in W'}\bigwedge_{i=1}^s (\sign(f_i(Y_1, \dots, Y_n))=a_i) 
     \end{equation}
     とすれば,主張は満たされる.

     次に,$m \geq 1$の場合を考える.
     一般性を失わずに,$f_1, \dots, f_s$が恒等的に$0$でないとしてよい.
     また,順番を並び替えて$f_s$の$X$についての次数が$m$であるとしてよい.
     このとき,多項式列の次数$f_1, \dots, f_s$の$X$についての次数がそれぞれ$m_1, \dots, m_s$とすれば,
     \begin{equation}
          f_i(X,Y_1, \dots, Y_n) = h_{i,m_i}(Y_1, \dots, Y_n)X^m_i + \dots + h_{i,0}(Y_1, \dots, Y_n), \quad i=1, \dots, s
     \end{equation}
     とかくことができる.ただし,各$i=1, \dots, m$に対して$h_{i,m_i}(Y_1, \dots, Y_n)$は恒等的に$0$でない.

     $m \geq 1$の場合の証明は,\cref{lemma:qe_lowering}を用いて多項式の$X$についての「次数を下げていく」ことで,
     最終的に$f_1, \dots, f_s \in \Z[Y_1, \dots, Y_n]$の場合に帰着させることで達成される.

     まず,「次数を下げていく」ことについてより詳しく述べる.
     非負整数列全体$\{(m_1, \dots, m_s) \mid s \geq 1, m_1, \dots, m_s\}$上の前順序関係
     \begin{equation}
          (m_1, \dots, m_s) \prec (m_1', \dots, m_t')
     \end{equation}
     を,ある非負整数$p$が存在し,$m_i$が$p$である$i=1, \dots, s$の数が,$m_j'$が$p$である$j=1, \dots, t$の数より少なく,
     かつ任意の$q>p$に対しては$m_i$が$q$である$i=1, \dots, s$の数と,$m_j'$が$q$である$j=1, \dots, t$の数が等しいことと定義する.

     この前順序関係のもとで,非負整列集合全体の中に無限降鎖列は存在しない事に注意する.
     すなわち,前順序関係$\prec$による降鎖列を際限なく取っていくと,いつかは$0$だけの列$(0,\dots, 0)$となる.

     多項式列$(f_1, \dots, f_s)$に対しては,$X$についての次数$(m_1, \dots, m_s)$を取ることで非負整数列に対応づく.
     これで,多項式列についての「次数下げ」を行っていくことができれば,いずれ$X$についての次数が全て$0$である場合に帰着され,証明が完了する.

     では,「次数下げ」の仕組みについて説明する.そのうえで重要なことは,次の主張である.  
     \begin{claim*}
          $ f , g \in \Z[X, Y_1, \dots, Y_n]$とし,それぞれの$X$についての次数が$m$, $l$であるとする.ただし,$m \geq l$とし,$g$は恒等的に$0$でないとする.
          \begin{align}
               f &= u_m(Y_1, \dots, Y_n)X^m + \dots + u_0(Y_1, \dots, Y_n), \quad u_m(Y_1, \dots, Y_n) \neq 0, \\
               g &= v_l(Y_1, \dots, Y_n)X^l + \dots + v_0(Y_1, \dots, Y_n), \quad v_l(Y_1, \dots, Y_n) \neq 0
          \end{align}
          とすると,$\Z(Y_1, \dots, Y_n)[X]$上$f$を$g$で割った余りを$r \in \Z(Y_1, \dots, Y_n)[X]$とすると,ある正整数$p>0$が存在し,
          \begin{equation}
               v_l(Y_1, \dots, Y_n)^p \cdot r(X,Y_1, \dots, Y_n) \in \Z[Y_1, \dots, Y_n][X]
          \end{equation}
          が成り立つ.また,この条件は,$p>m-l$であれば必ず満たされる.
     \end{claim*}
     この主張は,通常の割り算のアルゴリズムから従う.

     では,「次数下げ」の仕組みについて説明する.

     $y_1, \dots, y_n \in R$が,ある$i=1, \dots, s$に対して$h_{i, m_i}(y_1, \dots, y_n) = 0$となる場合,
     そのような$i$については$\tilde{f}_i := f_i - h_{i, m_i}X^{m_i}$とし,それ以外の$i$については$\tilde{f}_i := f_i$として得られる多項式列$(\tilde{f}_1, \dots, \tilde{f}_s)$
     について,
     \begin{align}
          &\SIGN_R(f_1(X,y_1, \dots, y_n), \dots, f_s(X,y_1, \dots, y_n)) \in W\\
          \iff &\SIGN_R(\tilde{f}_1(X,y_1, \dots, y_n), \dots, \tilde{f}_s(X,y_1, \dots, y_n)) \in W
     \end{align}
     である.

     また,$y_1, \dots, y_n \in R$が,全ての$i=1, \dots, s$に対して$h_{i, m_i}(y_1, \dots, y_n) \neq 0$となる場合,
     $r_1, \dots, r_s \in \Z(Y_1, \dots, Y_n)[X]$を,$f_s$をそれぞれ$f_1, \dots, f_{s-1}, f'_s$で割った余りとし,$g_1, \dots, g_s \in \Z[X,Y_1, \dots, Y_n]$を,
     \begin{equation}
          g_i = \begin{cases}
               r_i \cdot h_{i,m_i}^{2(m_s-m_i+1)} & i=1, \dots, s-1\\
               r_s \cdot (m_s h_{i,m_s})^{2\cdot2} & i=s
          \end{cases}
     \end{equation}
     と定めれば,\cref{lemma:qe_lowering}より,多項式列$(f_1, \dots, f_s', g_1, \dots, g_s)$について,
     \begin{align}
          &\SIGN_R(f_1(X,y_1, \dots, y_n), \dots, f_s(X,y_1, \dots, y_n)) \in W\\
          \iff &\SIGN_R(f_1(X,y_1, \dots, y_n), \dots, f_s'(X,y_1, \dots, y_n),g_1(X,y_1, \dots, y_n), \dots, g_s(X,y_1, \dots, y_n)) \in \phi^{-1}(W)
     \end{align}
     が成り立つ.

     いずれの場合も,多項式列の次数を下げることができている.よって,証明が完了した.
\end{proof}

\section{半代数的集合}
% ここに半代数的集合について書く.

\section{柱状代数分解}

$R$を実閉体とする.

集合$X$の部分集合族$\calS$が,
\begin{enumerate}
     \item $X = \bigcup_{S \in \calS} S$,
     \item 任意の$S_1, S_2 \in \calS$に対し,$S_1 \neq S_2$ならば$S_1 \cap S_2 = \emptyset$.
\end{enumerate}
を満たすとき,部分集合族$\calS$を$X$の分解(decomposition)と呼ぶ.

まず始めに,$R^n$の柱状代数分解を定義する.この定義は,参考文献\cite{MR2248869}による.

\begin{definition} \label{definition:cad}
     $i=1, \dots, n$に対し,$\calS_i$を$R^i$の分解とする.
     $\{\calS_i\}_{i=1}^n$が$R^n$の柱状代数分解(cylindrical algebraic decomposition)であるとは,
     以下の条件を満たすことをいう.
     \begin{enumerate}
          \item $S \in \calS_1$は,$R$上の点か,開区間かのいずれかである.
          \item $n\geq 2$の場合,任意の$i=1, \dots, k-1$と任意の$S \in \calS_i$に対し,
          有限個の連続な半代数的関数
          \begin{equation}
               \map{\xi_{S,1}< \dots <\xi_{S,l_S}}{S}{R}
          \end{equation}
          が存在し,以下を満たす.(ただし,$l_S$は0以上の自然数とする.)
          \begin{itemize}
               \item 各$j=1 \dots, l_S$に対し,$\xi_{S,j}$のグラフ
               \begin{equation}
                    \{(x,x_{i+1}) \mid x \in S, \xi_{S,j}(x)=x_{i+1} \} \subset R^{i+1}
               \end{equation}
               は,$\calS_{i+1}$の元である.
               \item $\xi_{S,0}=-\infty$, $\xi_{S,l_S+1}=\infty$とするとき,各$j=0, \dots, l_S$に対し,\label{cad_condition1}
               \begin{equation}
                    \{(x,x_{i+1}) \mid x \in S, \xi_{S,j}(x)<x_{i+1}<\xi_{S,j+1} \} \subset R^{i+1}
               \end{equation}
               は,$\calS_{i+1}$の元である.
          \end{itemize}
     \end{enumerate}
\end{definition}

定義より,任意の$k=1, \dots, n$に対して,$\{\calS_i\}_{i=1}^k$は$R^k$の柱状代数分解である.
逆に,$R^{n-1}$の柱状代数分解$\{\calS_i\}_{i=1}^{n-1}$が与えられているとき,各$S \in \calS_{n-1}$に対して
\cref{definition:cad}の条件\ref{cad_condition1}を満たすような有限個の連続半代数関数$\xi_{S,1}<\dots<\xi_{S,l_S}$が与えられれば,
$R^n$の柱状代数分解$\{\calS_i\}_{i=1}^n$を構成することができる.
このように,柱状代数分解は再帰的な構造である.

\begin{definition}
     $F \subset R[X_1, \dots, x_n]$を有限部分集合とする.
     \begin{enumerate}
          \item 部分集合$S \subset \R^n$が$F$符号不変($F$-invariant)であるとは,
          任意の$f \in F$及び任意の$x,y \in S$に対し$\sign(f(x))=\sign(f(y))$であることと定義する.
          \item $R^n$の柱状代数分解$\{\calS_i\}_{i=1}^n$が$F$に適合している(adapted to $F$)とは,
          任意の$C \in \calS_k$が$F$符号不変であることと定義する.
     \end{enumerate}
\end{definition}

この節の目的は,次の定理を示すことである.

\begin{theorem} \label{theorem:cad}
     任意の有限部分集合$F \subset R[X_1, \dots, x_n]$に対して,
     $F$に適合した$R^n$の柱状代数分解が存在する.
\end{theorem}

この定理は,柱状代数分解を再帰的に構成することで証明される.
まず初めに$R^1$の柱状代数分解$\{\calS_1\}$を与え,
それを基に順に$R^k$の柱状代数分解$\{\calS_i\}_{i=1}^k$を構成していく.(ただし$k=2, \dots, n$.)

問題になるのは,$F$に適合した$R^n$の柱状代数分解を構成することである.
以下の節では,各$k=1, \dots, n-1$で$R^k$の柱状代数分解$\{\calS_i\}_{i=1}^k$を構成するにあたり
どのような条件を満たす必要があるかを述べていく.

\subsection{描画可能}
$F$に適合した$R^n$の柱状代数分解$\{\calS_i\}_{i=1}^n$を$R^{n-1}$の柱状代数分解$\{\calS_i\}_{i=1}^n$から構成するとき,
各$S \in \calS_{n-1}$は次に定義する性質を満たしてほしい.

なお,次の定義は参考文献\cite{MR0403962}によって導入されたものである.

\begin{definition} 
     $F \subset R^n[X_1, \dots, X_n]$を有限部分集合とし,$S \subset R^{n-1}$を空でない半代数的連結な半代数的集合とする.
     $S$が$F$描画可能($F$-delineable)であるとは,$k$個の連続な半代数的関数$\map{\xi_1<\dots<\xi_k}{S}{R}$が存在し,次を満たすことと定義する.
     \begin{itemize}
          \item 任意の$ x \in S $に対し,
          \begin{equation}
               \{\xi_i(x)\}_{i=1}^k = \{y \in R \mid \prod_{f \in F'}f(x,y)=0\}
          \end{equation}
          である.ただし,$F' = \{f \in F \mid \text{$f$は$S$上恒等的に0でない}\}$とする.
          \item 各$i=1, \dots, k$および$f \in F'$に対し,多項式$f(x,X_n)$における根$\xi_i(x)$の重複度は,$x\in S$によらず一定.
     \end{itemize}
\end{definition}

$R^{n-1}$の柱状代数分解$\{\calS_i\}_{i=1}^n$で,
各$S \in \calS_{n-1}$が$F$描画可能であるとき,定義より得られる連続な半代数関数$\map{\xi_1<\dots<\xi_k}{S}{R}$を
各$S$に対して取ることにより,$F$適合な$R^n$の柱状代数分解$\{\calS_i\}_{i=1}^n$を得ることができる.

$F$描画可能であることの必要条件は,次のように与えられる.

\begin{proposition}\label{proposition:del}
     $F \subset R[X_1, \dots, X_n]$を有限部分集合とし,$S \subset \R^{n-1}$を半代数的連結な半代数的集合とする.
     さらに,以下を満たすとする.
     \begin{enumerate}
          \item 任意の$f \in F$に対し,$S$上$f$の$R[\sqrt{-1}]$上の根の数は重複度込みで一定である.
          \item 任意の$f \in F$に対し,$S$上$f$の$R[\sqrt{-1}]$上の相異なる根の数は一定である.
          \item 任意の$f, g \in F$に対し,$S$上$f$, $g$に共通する$R[\sqrt{-1}]$上の根の数は重複度込みで一定である.
     \end{enumerate}
     このとき,$S$は$F$描画可能である.
\end{proposition}

\cref{proposition:del}を示すために,次の二つの補題を用意する.

\begin{lemma}\label{lemma:del_1}
     $S \subset R^{n-1}$を空でない半代数的連結な半代数的集合部分集合とし,
     $f_1, f_2 \in R[x_1, \dots, x_n]$を$S$上恒等的に$0$でない多項式とする.
     $f_1, f_2$が次を満たすとする.
     \begin{enumerate}
          \item 各$i=1, 2$に対し,$S$上$f_i$の$R[\sqrt{-1}]$上の根の数は重複度込みで一定.
          \item 各$i=1, 2$に対し,$S$上$f_i$の$R[\sqrt{-1}]$上の相異なる根の数は一定.
          \item $S$上$f_1, f_2$の$R[\sqrt{-1}]$上の共通根の数は重複度込みで一定.
     \end{enumerate}
     このとき,$S$上$f_1, f_2$の相異なる$R[\sqrt{-1}]$上の共通根の数は一定である.
\end{lemma}
\begin{proof}
     簡単のため,$Y:=X_n$, $X:=(X_1, \dots, X_{n-1})$と表す.
     このとき,$f_1, f_2$は$f_1(X,Y), f_2(X,Y)$と表すことができる.
     $x \in S$に対して,$f_1(x,Y), f_2(x,Y)$の$R[\sqrt{-1}]$上の相異なる共通根の数を$N(x)$と定義する.
     すると,$\map{N}{S}{R}$は半代数的関数である.
     実際,$N$のグラフは,次のように書き表される.
     \footnote{$N$が半代数的関数であることは,$N$のグラフを以下のように書き表すことができるためである:
     \begin{equation}
          \{(x,z) \mid \bigvee_{n}(z = n \text{ and } \exists y_1, \dots, y_n \text{ s.t. } y_i \neq y_j( i\neq j) \text{ and } \forall y(f_1(x,y)=f_2(x,y) \leftrightarrow \bigvee_{i=1}^n (y = y_i)))\}
     \end{equation}
     ここで,論理和の$n$は,$0$から$\gcd(P,Q)$の$Y$に関する次数までをとる.
     \cref{theorem:Tarski}より,$\RCF$において量化記号消去可能であるから,
     $N$のグラフは半代数的である.
     }

     次の主張を示せば,$N$は$S$上の局所定数半代数的関数であるため,$N$は定数関数となり,証明が完了する.
     \begin{claim*}
          任意の$x \in S$に対し,$x$の開近傍$V$で,
          $S\cap V$上$f_1(X,Y), f_2(X,Y)$の$R[\sqrt{-1}]$上の相異なる共通根の数が一定となるようなものが存在する.
     \end{claim*}
     以降は主張を示す.
     \todo[inline]{
          この後は,方程式の解が係数に関して連続であることから示される.
          (実閉体の場合も同様に示すことができる,証明は逆関数定理を経由する.)}
\end{proof}

\begin{lemma}\label{lemma:del_2}
     $S \subset R^{n-1}$を半代数的連結な半代数的集合とし,     
     $A \in R[x_1, \dots, x_n]$を$S$上恒等的に$0$でない多項式とする.
     さらに,次を満たすとする.
     \begin{itemize}
          \item $S$上$A$の$R[\sqrt{-1}]$上の根の数は重複度込みで一定.
          \item $S$上$A$の$R[\sqrt{-1}]$上の相異なる根の数は一定.
     \end{itemize}
     このとき,$S$は$\{A\}$描画可能である.
\end{lemma}

\begin{proof}
     \todo[inline]{証明が書かれていないので,証明を書く.}
\end{proof}

\begin{proof}[{\bf \cref{proposition:del}の証明}]
任意の$f \in F$は$S$上恒等的に$0$でないとしてよい.

\cref{lemma:del_2}より,各$f \in F$に対して,$S$は$\{f\}$描画可能である.
よって,$S$上の連続関数$\alpha_{1,f}(a) < \dots \alpha_{n_f, f}(a)$を,各$a \in S$で$f(a)(x) \in \R[x]$の解であるようにとれる.

次の主張が成り立てばよい.

\begin{claim*}
$f, g \in F$が,$f \neq g$であるとする.
ある$a \in S$において$\alpha_{k,f}(a) = \alpha_{l,g}(a)$であるならば,任意の$a \in S$に対して$\alpha_{k,f}(a) = \alpha_{l,g}(a)$である.
\end{claim*}

この主張は,$S$が半代数的連結であることと,\cref{lemma:del_1}から従う.よって,命題が示された.
\end{proof}

\begin{corollary}\label{corollary:del}
$S \subset \R^{n-1}$を弧状連結部分集合とし,$F \subset \R[x_1,\dots, x_n]$を有限部分集合とする.
次が成り立つとき,$S$は$F$描画可能である.
\begin{itemize}
\item 任意の$f \in F$に対し,$\deg(f(a))$が一定($a \in S$).
\item 任意の$f \in F$に対し,$\deg(\gcd(f(a), \frac{\partial f}{\partial x_n}(a)))$が一定($a \in S$).
\item 任意の$f, g \in F$に対し,$\deg(\gcd(f(a), g(a)))$が一定($a \in S$).
\end{itemize}
\end{corollary}

よって,$F$符号不変な$\R^n$の分割を与えるには,\cref{corollary:del}の条件を満たすような$\R^{n-1}$の分割を構成すればよい.
一つ目の条件を満たすような$\R^{n-1}$の分割を与えるには,各$f \in F$について,$x_n$係数が符号不変になるような分割を構成すればよい.
しかし,二つ目と三つ目の条件を満たすような$\R^{n-1}$の分割を与えるのは少し難しい.
なぜなら,多項式$f, g \in \R[x_1, \dots, x_n]$について,$a \in \R^{n-1}$を固定したとき,$\gcd(f,g)(a)$と$\gcd(f(a),g(a))$は必ずしも等しくないからである.

よって,二つ目と三つ目の条件も満たすような$\R^{n-1}$の分割を与えることができるように次の節で準備をする.

\subsection{主部分終結式係数(Principal Subresultant Coefficient)}

\begin{definition}
$\mathrm{R}$を可換環とし,$A(x), B(x) \in \mathrm{R}[x]$ を,$\deg A(x) = m$, $\deg B(x) = n$ とする.ただし,$\deg 0 = 0$と解釈する.

$j = 0, \dots, \min\{n, m\}$に対し,多項式$A(x)$, $B(x)$の$j$次部分終結式$S_j(A, B)$を次のように定義する.
\begin{align}
A(x) = a_m x^m + \dots + a_1 x + a_0, \\
B(x) = b_n x^n + \dots + b_1 x + b_0 
\end{align}
として,$j = 0, \dots, \min\{n,m\}$に対し,
\begin{align}
M_j = 
\begin{pmatrix}
a_m & a_{m-1} & \cdots & a_1 & a_0 &    &  \\
     &  a_m     & \cdots & a_2 & a_1& a_0 &  \\
     &   & \ddots &  & & \\
b_n & b_{n-1} & \cdots & b_1 & b_0 &    & \\
     &  b_n     & \cdots & b_2 & b_1& b_0 & \\
     &   & \ddots &  & & 
\end{pmatrix}
\in \mathrm{M}_{m+n-2j, m+n-j}(\mathrm{R})
\end{align}
とする.ここで,行列の空白部分はすべて$0$であり,また,行列の上側は$n-j$行,行列の下側は$m-j$行である.

また,$j = 0, \dots, \min\{m,n\}$, $i = 0, \dots, j$に対し,
\begin{align}
M_{j,i} = (\text{$M_j$の第$1$列}, \text{$M_j$の第$2$列}, \dots ,\text{$M_j$の第$m+n-2j-1$列}, \text{$M_j$の第$m+n-i-j$列})
\in \mathrm{M}_{m+n-2j, m+n-2j}(\mathrm{R})
\end{align}
とする.$j = 0, \dots, \min\{n, m\}$に対し,
\begin{align}
S_j(A, B) = \sum_{i=0}^j \det M_{j, i} \cdot x^i 
\end{align}
を多項式$A(x)$, $B(x)$の$j$次部分終結式$S_j(A, B)$という.

この$j$次部分終結式の先頭項係数を,$\psc_j(A,B)$とかき,多項式$A(x)$, $B(x)$の$j$次主部分終結式係数という.
\end{definition}


主部分終結式係数は,多項式$A(x)$, $B(x)$の係数,及び次数に依存して定まる.

\begin{remark}
$m$, $n$のどちらかが$0$のとき,
$S_0(A,B) = 0$, $\psc_0(A,B) = 0$とする.

また,$\psc_0(A,B)$は,多項式$A(x)$, $B(x)$の終結式に一致する.
\end{remark}

\begin{proposition}\label{proposition:psc}
$\mathrm{R}$を体とし,$A(x), B(x) \in \mathrm{R}[x] \setminus \{0\}$とすると,次が成立する.
\begin{align}
\deg(\gcd(A, B)) = \min \{ j  \in \{0,1, \dots, \min\{n,m\}\}\mid \psc_j(A,B) \neq 0\}
\end{align}
\end{proposition}

\begin{proof}
\todo[inline]{後で書く.方針: 部分終結式がユークリッドの互除法で出てくる多項式の列の定数倍になることが分かる.}
\end{proof}

\subsection{符号不変な分割の存在}
ここでは\cref{theorem:cad}を証明する.
そのために,与えられた有限部分集合$F \subset R[X_1, \dots, X_n]$に対して,
帰納的な議論で$R^n$の$F$適合な柱状代数分解の存在を示す.

まず,帰納的な議論のために,次のような記号を定義する.

\begin{definition}
$F \subset R[X_1, \dots, X_n]$を有限部分集合とする.ただし,$n \geq 2$とする.
$\PROJ(F) \subset R[X_1, \dots, X_{n-1}]$を,次のように定める.

まず,$B(F) := \{ \mathrm{red}^k(f) \mid f \in F, k=1, \dots, \deg(f) \}$とする.ただし,$\mathrm{red}(f) = f - \mathrm{LT}(f; X_n)$である.
次に,
\begin{align}
	\PROJ_1(F) &:= \{\mathrm{LC}(f;X_n) \mid f \in B\},\\
	\PROJ_2(F) &:= \{\psc_j(f, \frac{\partial f}{\partial X_n}; X_n) \mid f \in B, j =0, \dots, \deg(\frac{\partial f}{\partial X_n};X_n) \},\\
	\PROJ_3(F) &:= \{\psc_j(f,g;X_n) \mid f,g \in B, j = 0, \dots, \min\{\deg(f;X_n), \deg(g;X_n)\}\}
\end{align}
とし,以上を用いて
\begin{align}
	\PROJ(F) &:= \PROJ_1(F) \cup \PROJ_2(F) \cup \PROJ_3(F)
\end{align}
と定める.
\end{definition}

この記号のもとで,次の系が成り立つ.

\begin{corollary}\label{corollary:inv-deline}
$F \subset \R[X_1, \dots, X_n]$を有限部分集合とし,$S \subset \R^{n-1}$を半代数連結な半代数的集合とする.

$S$が$\PROJ(F)$符号不変ならば,$S$は$F$描画可能である.
\end{corollary}

\begin{proof}
\cref{proposition:del}及び\cref{proposition:psc}から従う.
\end{proof}

それでは,\cref{definition:cad}の証明を述べる.

\begin{proof}[\cref{theorem:cad}の証明]
     $n$についての帰納法によって示す.

     $n=1$の場合,$F$の$R$上の根を$a_1 < \dots, a_k$とする.
     $a_0 = -\infty$, $a_{k+1} = \infty$とすれば,
     \begin{equation}
          \calS_1 = \{(a_i, a_{i+1})\}_{i=0}^k \cup \{a_i\}_{i=1}^k
     \end{equation}
     が$F$-適合な柱状代数分解を与える.よって,定理は成り立つ.

     $n\geq 2$の場合,$n-1$までで主張が成り立つとする.
     このとき,帰納法の仮定により$\PROJ(F)$適合な$R^{n-1}$の柱状代数分解を$\{\calS_i\}_{i=1}^{n-1}$がとれる.

     ここで,各$S \in \calS_{n-1}$は$\PROJ(F)$不変であるから,\cref{corollary:inv-deline}より,$S$は$F$描画可能である.

     従って,各$S \in \calS_{n-1}$に対して,有限個の半代数的連続関数$\map{\xi_{S,1}< \dots < \xi_{S,l_S}}{S}{R}$が存在し,
     \begin{itemize}
          \item 任意の$ x \in S $に対し,
          \begin{equation}
               \{\xi_{S,i}\}_{i=1}^{l_S} = \{y \in R \mid \prod_{f \in F'}f(x,y)=0\}
          \end{equation}
          である.ただし,$F' = \{f \in F \mid \text{$f$は$S$上恒等的に0でない}\}$とする.
          \item 各$i=1, \dots, l_S$および$f \in F'$に対し,多項式$f(x,X_n)$における根$\xi_{S,i}(x)$の重複度は,$x\in S$によらず一定.
     \end{itemize}
     であるようにとれる.
     また,$\xi_{S,0} := -\infty$, $\xi_{S,l_S} := \infty$としておく.

     このとき,
     \begin{align}
          C_{S,2i} &:= \{(x,y) \mid  x \in S, \xi_{S,i}(x) = y \} \quad i = 1,\dots, l_S,\\
          C_{S,2i+1} &:= \{(x,y) \mid x \in S, \xi_{S,i}(x)<y<\xi_{S,i+1}(x) \} \quad i = 0,1, \dots, l_S 
     \end{align}
     とし,
     \begin{equation}
          \calS_n := \{C_{S,i} \mid S \in \calS_{n-1}, i=1, \dots, 2l_S+1\}
     \end{equation}
     とすれば,$\{\calS_i\}_{i=1}^n$は,$R^n$の$F$適合な柱状代数分解を与える.

     よって,帰納法により示された.
\end{proof}

% \begin{definition}
% $ F \subset \mathbb{R}[x_1,\dots,x_n] $ を有限部分集合とする.$ S \subset \mathbb{R}^{n-1} $が$ F $描画可能とする.
% このとき,$ S $上の$ F $の解を$ f_1(x)< \dots <f_k(x) $とし,$ f_0(x) := -\infty $ , $ f_{k+1}(x) := \infty $とするとき,
% \begin{align}
%   C_{2i} &:= \{(x,y) \mid  x \in S, f_i(x) = y \} \quad i = 1,\dots, k,\\
%   C_{2i+1} &:= \{(x,y) \mid x \in S, f_{i}(x)<y<f_{i+1}(x) \} \quad i = 0,1, \dots, k 
% \end{align}

% とすれば,$\{C_j\}_{j=1}^{2k+1}$は$ S \times \mathbb{R} $の$F$符号不変な分割を与える.
% この$ S \times \mathbb{R} $の分割$ \{C_j\}_{j=1}^{2k+1} $を,$ S $の持ち上げといい,$ \mathfrak{L}(S) $と書く.

% また,$\mathfrak{D}$が$\R^{n-1}$の分割であるとき,$\mathfrak{L}(\mathfrak{D}) := \bigcup_{D \in \mathfrak{D}}\mathfrak{L}(D)$を$\mathfrak{D}$の持ち上げという.
% \end{definition}

\section{柱状代数分解による量化記号消去}
有限部分集合$F \subset R[x_1, \dots, x_n]$が与えられたとき,
前節では$\R^n$の$F$適合な柱状代数分解$\mathfrak{D}$が存在することを示した.
この節では,柱状代数分解による量化記号消去のアルゴリズムについて示す.

冠頭標準形の論理式
\begin{equation}
     \Qua_1 X_1 \dots \Qua_1 X_k \phi(Y_1, \dots, Y_l, X_1, \dots, X_k)
\end{equation}
が与えられたとき,
$R^{k+l}$の柱状代数分解$\{\calS_i\}_{i=1}^{k+l}$を用いて量化記号消去を行うことを考える.

そのためには,$\calS_l$に属する各$S \in \calS_l$の定義式が与えられている必要がある.
しかし,前節で示した柱状代数分解の構成では,各$S \in \calS_l$の定義式は与えられない.

そのため,柱状代数分解のアルゴリズムを改良し,
$\{\calS_i\}_{i=1}^l$に属する各$S \in \calS_i$の定義式を与えるようにする必要がある.
そのために,まずは増補射影について定義する.

\subsection{増補射影}
     多項式の有限部分集合$F \subset R[X]$が,微分について閉じている(closed under differentiation)
     とは,任意の$f \in F$に対して,$f' \in F$または$f'=0$となることと定義する.

\begin{theorem}[Thomの補題]
     $F = \{f_1, \dots, f_s\} \subset R[X]$を,微分について閉じた有限部分集合とする.
     このとき, 任意の$\map{\sigma}{\{1, \dots, s\}}{\{-1,0,1\}}$に対し,
     \begin{equation}
          A_\sigma := \{x \in R \mid \text{任意の$i=1, \dots, s$に対し,} \sign(f_i(x)) = \sigma(i)\}
     \end{equation}
     と定義する.
     このとき,$A_\sigma$は空集合,一点からなる集合,開区間のいずれかである.
\end{theorem}

\begin{proof}
     $s$に関する帰納法で示す.
     $s = 0$のときは明らかである.
     
     次に,$s$において主張が成り立つとする.
     $F = \{f_1, \dots, f_{s+1}\}$とする.ここで,順番を並び替えて$f_{s+1}$の次数が最大であるとする.

     任意に$\map{\sigma}{\{1, \dots, s, s+1\}}{\{-1,0,1\}}$をとる.
     $\map{\sigma_0}{\{1, \dots, s\}}{\{-1,0,1\}}$を$\sigma$の制限,すなわち
     \begin{equation}
          \sigma_0(i) := \sigma(i), \quad i=1, \dots, s
     \end{equation}
     とする.

     このとき,$F_0 := \{f_1, \dots, f_{s}\}$は微分について閉じているので,帰納法の仮定より
     \begin{equation}
          A_{\sigma_0}:=\{x \in R \mid \text{任意の$i=1, \dots, s$に対し,} \sign(f_i(x)) = \sigma_0(i)\}
     \end{equation}
     は,空集合,一点からなる集合,開区間のいずれかである.

     $A_\sigma = A_{\sigma_0} \cap \{x \in R \mid \sign(f_{s+1}(x)) = \sigma(s+1)\}$であるから,
     $A_{\sigma_0}$が空集合,あるいは一点からなる集合の場合には,$A_\sigma$は空集合か一点からなる集合になるので主張は成り立つ.
     したがって,$A_{\sigma_0}$が開区間である場合を考える.
     $f_{s+1}$の微分$f_{s+1}'$は,$f_1, \dots, f_s$のいずれかであるから,$A_{\sigma_0}$上符号は一定である.
     $f_{s+1}'$の符号が$A_{\sigma_0}$上常に$0$である場合,$f_{s+1}$は$A_{\sigma_0}$上定数であるから,主張は成り立つ.
     $f_{s+1}'$の符号が$A_{\sigma_0}$上常に$-1$(あるいは$1$)の場合を考える.
     このとき$f$は$A_{\sigma_0}$上狭義単調増加(あるいは狭義単調減少)であるから,$A_{\sigma}$は空集合,一点からなる集合,開区間のいずれかである.

     以上より,主張は示された.
\end{proof}

\begin{definition}[増補射影]
     後で書く.

\end{definition}

増補射影を用いることで,帰納的に定義式が分かっているような分割を得ることができる.

\begin{theorem}
     $F \subset R[X_1, \dots, X_n]$を有限部分集合とし,$n \geq 2$とする.
     $S \subset R^{n-1}$を,半代数的連結な半代数的部分集合とする.
     $S$が$\APROJ(F)$不変ならば,$S$は$\der(F)$描画可能である.

     特に,$\map{\sigma}{\{1,\dots, s\}}{\{-1, 0, 1\}}$と$g_1, \dots g_s \in R[X_1, \dots, X_{n-1}]$を用いて,
     \begin{equation}
          S = \{(x_1, \dots, x_{n-1}) \in R^{n-1} \mid \sign(g_i(x_1, \dots, x_{n-1}))= \sigma_i\}
     \end{equation}
     であるならば,
     $\der(F)$による$S$の持ち上げ$\calL(S)$の定義式は,$\{g_1, \dots,g_s\} \cup \der(F)$で記述できる.
\end{theorem}

\subsection{柱状代数分解による量化記号消去のアルゴリズム}

柱状代数分解による量化記号消去のアルゴリズムは,次のように述べられる.

\begin{algorithm}
     \caption{QE}
     \begin{algorithmic}[1]
     \REQUIRE 
     \ENSURE 
     \STATE pass
     \end{algorithmic}
\end{algorithm}

% \begin{thebibliography}{99}
%      \bibitem{Arai} 新井敏康,『数学基礎論』(2016).
%      \bibitem{Itai} 板井昌典,『モデル理論』(2023).
%      \bibitem{Anai} 穴井宏和,横山和弘,『QEの計算アルゴリズムとその応用 数式処理による最適化』(2011).
%      \bibitem{Artin} Artin, E., Schreier, O. Algebraische Konstruktion reeller Körper. Abh.Math.Semin.Univ.Hambg. 5, 85–99 (1927). https://doi.org/10.1007/BF02952512
%      \bibitem{Bochnak}Jacek Bochnak, Michel Coste, Marie-Fran\c{c}oise Roy, Real Algebraic Geometry, Springer(1998).
%      \bibitem{Basu} Saugata Basu, Richard Pollack, Marie-Fran\c{c}oise Roy, Algorithms in Real Algebraic Geometry(2003).
%      \bibitem{Collins} George E. Collins, Quantifier elimination for real closed fields by cylindrical algebraic decomposition(1975).
% \end{thebibliography}


\bibliographystyle{unsrt}
\bibliography{references.bib}
\end{document}