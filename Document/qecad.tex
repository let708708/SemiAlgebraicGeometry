\documentclass[uplatex, dvipdfmx]{jsarticle}

\usepackage{amsmath,amssymb}
\usepackage{amsthm}
\usepackage{algorithmic}
\usepackage{algorithm}
\usepackage{cleveref}

\makeatletter
\renewenvironment{proof}[1][\proofname]{\par
  \pushQED{\qed}%
  \normalfont \topsep6\p@\@plus6\p@\relax
  \trivlist
  \item\relax
  {\bfseries
  #1\@addpunct{.}}\hspace\labelsep\ignorespaces
}{%
  \popQED\endtrivlist\@endpefalse
}
\makeatother

\newcommand{\R}{\mathbb{R}}
\newcommand{\Q}{\mathbb{Q}}
\newcommand{\Ralg}{\mathbb{R}_\mathrm{alg}}
\newcommand{\Z}{\mathbb{Z}}
\newcommand{\Qua}{\mathfrak{Q}}
\newcommand{\M}{\mathfrak{M}}
\newcommand{\calL}{\mathcal{L}}
\newcommand{\calS}{\mathcal{S}}
\newcommand{\calP}{\mathcal{P}}
\newcommand{\defiff}{ :\Leftrightarrow}
\newcommand{\RCF}{\mathrm{RCF}}
\newcommand{\Var}{\mathrm{Var}}
\newcommand{\psc}{\mathrm{psc}}
\newcommand{\PROJ}{\mathrm{PROJ}}
\newcommand{\APROJ}{\mathrm{APROJ}}
\newcommand{\der}{\mathrm{der}}
\newcommand{\sign}{\mathrm{sign}}
\newcommand{\SIGN}{\mathrm{SIGN}}
\newcommand{\map}[3]{{#1}:{#2}\rightarrow{#3}}

\theoremstyle{definition}
\newtheorem{definition}{定義}[section]
\crefname{definition}{定義}{定義}%
\newtheorem{proposition}{命題}[section]
\crefname{proposition}{命題}{命題}%
\newtheorem{lemma}{補題}[section]
\crefname{lemma}{補題}{補題}%
\newtheorem{theorem}{定理}[section]
\crefname{theorem}{定理}{定理}%
\newtheorem{corollary}{系}[section]
\crefname{corollary}{系}{系}%S
\newtheorem*{claim*}{主張}
\newtheorem{remark}{注意}[section]
\newtheorem{example}{例}[section]

\renewcommand{\proofname}{\textbf{証明}}

\begin{document}

\title{実閉体上の柱状代数分解による量化記号消去}
\author{富永 直弥}
\maketitle

\section{基礎的な定義}
% コンピュータによって「具体的な問題の」量化記号消去を行うというモチベーションで議論を進める.
% すると実数体を扱えないことから,実数体の公理を抜き出す動機が生まれそう?

\subsection{実閉体}
% この節では,Artin-Schreierにより導入された実閉体の理論を復習する.
% 実閉体の概念はArtinが1927年にHilbertの第17問題を解くのにあたり導入された.

% また,この説では参考文献\cite{Arai}に従い定義を書く.

\begin{definition}[順序体と実体の定義]
     体$F$と$F$上の全順序構造$<$の組$(F,<)$が次の性質を満たすとき,$(F,<)$を順序体(ordered field)という.
     \begin{enumerate}
          \item 任意の$x,y,z\in F$に対し,$x \leq y$ならば$x + z \leq y + z$である.
          \item 任意の$x,y \in F$に対し,$x \geq 0$, $y \geq 0$ならば$x \cdot y \geq 0$である.
     \end{enumerate}

     また,体$F$が,任意の有限個の$F$の元$x_1, \dots, x_n \in F$について, $-1 \neq \sum_{i=1}^k x_i^2$を満たすとき,実体(real field)であるという.
\end{definition}

順序体と実体について,次の命題が成り立つ.

\begin{proposition}
     $F$を体とする.$F$が実体であることと,$F$が順序体となるような全順序構造$<$を持つことは同値である.
\end{proposition}

例えば,実数体$\R$や有理数体$\Q$は,通常の全順序構造を入れることにより順序体となる.したがって,$\R$および$\Q$は実体である.

\begin{definition}[実閉体の定義]
     実体$F$が実閉体(real closed field)であるとは,$F$の代数拡大で実体になるものは$F$自身に限ることである.
\end{definition}

\begin{theorem}[E. Artin]
     順序体$(F,<)$は,次の二つの性質を満たせば実閉体である.
     \begin{enumerate}
          \item 任意の$x \in F$に対し, $x>0$ならば$x = y^2$を満たす$y \in F$が存在する.
          \item 奇数次数の$F$係数$1$変数多項式は,$F$上に少なくとも$1$つの解を持つ.
     \end{enumerate}
\end{theorem}

この定理により,実閉体の理論は一階述語論理で公理化することができる.

例えば,実数体$\R$は実閉体である.
また,代数的数全体(すなわち, 有理数体$\Q$の代数的閉包)$\bar{\Q}$と実数体$\R$の共通部分$\Ralg := \bar{\Q} \cap \R$は実閉体である.
$\Ralg$に属する元を実代数的数という.

\subsection{一階述語論理}
一階述語論理式(first-order logic)は,
変数記号,論理記号 ($=, \lnot, \lor, \land, \rightarrow$), 量化記号($\forall, \exists$)
と,言語$L$で記述される.

ここで,言語$L$とは,述語記号,関数記号,定数記号からなる集合である.
ただし,各述語記号$R \in L$は,何個の変数を受け取るか定まっているものとし,
同様に各関数記号$f \in L$は,何個の変数を受け取るか定まっているものとする.
$n$個の変数を受け取る述語記号を$n$変数述語記号といい,$n$個の変数を受け取る関数記号を$n$変数関数記号という.

\begin{example}[順序環の言語]
     次の記号からなる言語$L_\mathrm{OR}$を,順序環の言語という.
     \begin{itemize}
          \item 2変数述語記号 $<$,
          \item 2変数関数記号 $+$, $\cdot$,
          \item 定数記号 $1$, $0$.
     \end{itemize}
\end{example}

言語$L$が定義されると,順に項,原子論理式,論理式が順に定義される.

\begin{definition}[項の定義]
     順序環の言語$L$に対して,項(term)が帰納的に定義される.
     \begin{enumerate}
          \item 変数記号,定数記号は項である.
          \item $t_1, \dots, t_n$がいずれも項とする.任意の$n$変数関数記号$f \in L$に対し,$f(t_1, \dots, t_n)$は項である.
          \item 以上の規則により定まる記号列のみが項である.
     \end{enumerate}
\end{definition}

\begin{definition}[論理式の定義]
     言語$L$に対して,原子論理式(atomic formula)を,次のいずれかの形をした記号列で定義する.
     \begin{enumerate}
          \item $t_1, t_2$を項として,$(t_1=t_2)$.
          \item $R \in L$を$n$変数述語記号,$t_1, \dots, t_n$を項として,$(R(t_1, \dots, t_n))$.
     \end{enumerate}
     
     また,原子論理式$\phi$またはその否定$(\lnot \phi)$を,リテラル(literal)とよぶ.
     さらに,言語$L$に対して,論理式(formula)が帰納的に定義される.
     \begin{enumerate}
          \item 原子論理式は論理式である.
          \item $\phi$が論理式であるとすると,$(\lnot \phi)$は論理式である.
          \item $\phi$,$\psi$が論理式であるとすると,$(\phi \lor \psi)$, $(\phi \land \psi)$, $(\phi \rightarrow \psi)$はいずれも論理式である.
          \item $\phi$が論理式であるとすると,任意の変数記号$x$に対して$(\forall x \phi)$, $(\exists x \phi)$は論理式である.
          \item 以上の規則により定まる記号列のみが論理式である.
     \end{enumerate}
\end{definition}

論理式$\phi := \forall x (x^2 + y > 0)$について観察する.ここで,$x,y$はいずれも変数記号である.

これは,「全ての$x$に対して$x^2 + y > 0$が成り立つ」というように読むことができるが,$y$の値は自由に設定することができ,$y$の値が定まらない限りは真であるか偽であるかははっきりしない.
この$y$のように,自由な状態にある変数記号を論理式$\phi$の自由変数といい,一方$x$のように自由な状態にない変数記号を論理式$\phi$の束縛変数という.

厳密には,次のように定義する.

\begin{definition}[自由変数の定義]
     言語$L$を一つとる.
     項$t$に登場する全ての変数記号を$\Var(t)$で表す.

     論理式$\phi$の自由変数(free variable)全体$\Var(\phi)$を,次のように論理記号及び量化記号の数に関して帰納的に定義する.
     \begin{enumerate}
          \item $t_1, t_2$を項とするとき,$\Var((t_1=t_2)):=\Var(t_1)\cup\Var(t_2)$と定義する.
          \item $R \in L$を$n$変数述語記号,$t_1, \dots, t_n$を項とするとき,$\Var(R(t_1, \dots, t_n)):= \bigcup_{i=1}^n \Var(t_i)$と定義する.
          \item 論理式$\phi$に対して$\Var(\phi)$が定義されているとき,$\Var((\lnot \phi)):=\Var(\phi)$と定義する.
          \item 論理式$\phi, \psi$に対して$\Var(\phi), \Var(\psi)$が定義されているとき,
          $\Var((\phi \lor \psi)):=\Var(\phi)\cup\Var(\psi)$,
          $\Var((\phi \land \psi)):=\Var(\phi)\cup\Var(\psi)$,
          $\Var((\phi \rightarrow \psi)):=\Var(\phi)\cup\Var(\psi)$
          と定義する.
          \item 論理式$\phi$に対して,$\Var(\phi)$が定義されているとき,変数記号$x$に対して
          $\Var((\forall x\phi)):=\Var(\phi) \setminus \{x\}$,
          $\Var((\exists x\phi)):=\Var(\phi) \setminus \{x\}$
          と定義する.
     \end{enumerate}

     論理式$\phi$に登場する自由変数が高々$x_1, \dots, x_n$である事を明示したいとき,$\phi(x_1, \dots, x_n)$と書く.ただし,登場しない自由変数があっても良い.

     項$t$が自由変数を持たないとき,すなわち$\Var(t)=\emptyset$であるとき,$t$を閉項(closed term)という.
     また,論理式$\phi$が自由変数を持たないとき,すなわち$\Var(\phi)=\emptyset$であるとき,$\phi$を閉論理式(closed formula)という.
\end{definition}

\begin{definition}[理論]
     言語$L$に対して,閉論理式からなる集合を理論(theory)という.
\end{definition}

\begin{example}[実閉体の理論]
     順序環の言語$L_\mathrm{OR}=\{<, +, \cdot, 1, 0\}$上の実閉体の理論$\RCF$を,次のように定義する.
     
     \begin{itemize}
          \item 体の公理
          \begin{enumerate}
               \item $\forall x \forall y(x + y = y + x)$
               \item $\forall x \forall y \forall z((x + y) + z = x + (y + z))$
               \item $\forall x (x + 0 = x)$
               \item $\forall x \exists y (x + y = 0)$
               \item $\forall x \forall y(x \cdot y = y \cdot x)$
               \item $\forall x \forall y \forall z((x \cdot y) \cdot z = x \cdot (y \cdot z))$
               \item $\forall x (x \cdot 1 = x)$
               \item $\forall x \exists y ((x=0) \lor (x \cdot y = 1))$
               \item $\lnot(0 = 1)$
          \end{enumerate}
          \item 全順序の公理
          \begin{enumerate}
               \item $\forall x (\lnot(x<x))$
               \item $\forall x \forall y \forall z((x < y \land y < z) \rightarrow x < z)$
               \item $\forall x \forall y (\lnot(x = y) \rightarrow (x<y \lor y<x))$
          \end{enumerate}
          \item 順序体の公理
          \begin{enumerate}
               \item $\forall x \forall y \forall z (x \leq y \rightarrow x + z \leq y + z)$
               \item $\forall x \forall y ((0 < x \land 0 < y) \rightarrow 0 < x \cdot y )$
          \end{enumerate}
          \item 実数体の公理
          \begin{enumerate}
               \item $\forall x \exists y (0<x \rightarrow x=y^2)$
               \item $\forall a_{2n-1} \dots \forall a_1 \forall a_0 \exists x(\lnot(a_{2n-1} = 0) \land (a_{2n-1}\cdot x^{2n-1} + \dots a_1 \cdot x + a_0 = 0)) \quad (n=1,2,\dots)$ 
          \end{enumerate}
     \end{itemize}

     ただし,項$t_1, t_2$に対し,$t_1 \leq t_2$は$(t_1 < t_2)\lor(t_1 = t_2)$の省略記号とする.
     また,項$t$と$n=1,2,\dots$に対し,$t^n$は,$t^n:=t \cdot t^{n-1}$, $t^1:=t$により帰納的に定義された省略記号とする.
\end{example}

理論$\RCF$の閉論理式をすべて満たすような数学的構造を,実閉体とよびたい.
このような定義を正当化させるには,この数学的構造が,言語$L_\mathrm{OR}$で書かれた閉論理式を真か偽か判定できることが必要である.
そのために,まず言語を解釈する数学的構造を定義する.

\begin{definition}[構造の定義]
     言語$L$の構造(structure)とは,空でない集合$M$と,次のようなデータの組$\M=(M,\dots)$である.
     \begin{itemize}
          \item 各$n$変数述語記号$R \in L$に対応する部分集合$R^\M \subset M^n$.
          \item 各$n$変数関数記号$f \in L$に対応する写像$\map{f^\M}{M^n}{M}$.
          \item 各定数記号$c \in L$に対応する元$c^\M \in M$.
     \end{itemize}
\end{definition}

言語$L$の構造$\M$が与えられたとき,
言語$L$上の論理式$\phi$が構造$\M$で成り立つかどうか,すなわち充足関係があるかどうかを定義できる.

\begin{definition}
     言語$L$上の構造$\M = (M,\dots)$をとる.
     言語$L$に形式的な定数記号$\{c_m \mid m \in M\}$を追加した言語を$L(\M)$とする.
     ここで,各$m \in M$に対し, $c_m^{\M}:=m$とすることで,
     $\M$は言語$L(\M)$の構造に拡張できる.
     \begin{itemize}
          \item $L(\M)$上の閉項$t$の解釈$t^\M$を帰納的に定義する.
          \begin{enumerate}
               \item $f$を$n$変数関数記号とし, 閉項$t_1$, \dots $t_n$のそれぞれの解釈$t_1^\M,\dots,t_n^\M$が定まっているとする.
               このとき,$(f(t_1,\dots,t_n))^\M:=f^\M(t_1^\M,\dots,t_m^\M)$と定める.
          \end{enumerate}
          \item $L(\M)$上の閉論理式$\phi$に対し, 充足関係(satisfaction relation)があること($\M \models \phi$とかく)を帰納的に定義する.
          \begin{enumerate}
               \item $t_1,t_2$を$L(\M)$上の閉項とするとき, 
               \[
                    \M \models (t_1 = t_2) \defiff t_1^\M = t_2^\M,
               \]
               \item $t_1,\dots t_n$を$L(\M)$上の閉項,$R$を$n$変数述語記号とするとき,
               \[
                    \M \models (R(t_1, \dots, t_n)) \defiff (t_1^\M, \dots, t_n^\M) \in R^\M,
               \]
               \item $L(\M)$上の閉論理式$\phi$に対し
               \[
                    \M \models (\lnot \phi) \defiff \M \not\models \phi,
               \]
               \item $L(\M)$上の閉論理式$\phi, \psi$に対し
               \begin{align*}
                    &\M \models (\phi \lor \psi) \defiff \M \models \phi \text{または} \M \models \psi,\\
                    &\M \models (\phi \land \psi) \defiff \M \models \phi \text{かつ} \M \models \psi,\\
                    &\M \models (\phi \rightarrow \psi) \defiff \M \not\models \phi \text{または} \M \models \psi,
               \end{align*}
               \item $L(\M)$上の変数記号が高々$x$のみである論理式$\phi(x)$に対し
               \begin{align*}
                    &\M \models (\forall x \phi(x)) \defiff \text{任意の$m \in M$に対し}\M \models \phi(c_m),\\
                    &\M \models (\exists x \phi(x)) \defiff \text{ある$m \in M$に対し}\M \models\phi(c_m).
               \end{align*}
          \end{enumerate}
     \end{itemize}
\end{definition}

\begin{definition}
     $L$を言語とし,$T$を$L$上の理論とする.
     $L$上の構造$\M$が任意の$\phi \in T$に対して$\M\models \phi$であるとき,構造$\M$を$T$のモデル(model)といい,$\M \models T$とかく.
\end{definition}

言語$L_\mathrm{OR}$上の構造$\R=(\R,<,+,\cdot,1,0)$や$\Ralg=(\Ralg,<,+,\cdot,1,0)$は,いずれも実閉体である.したがって,
\[
     \R \models \RCF, \quad \Ralg \models \RCF
\]
であり,構造$\R$,$\Ralg$はいずれも実閉体の理論$\RCF$のモデルである.

\section{量化記号消去}
% この節では,実閉体の理論$\RCF$が量化記号消去可能であることについて述べる.

\begin{example}
     言語$L_\mathrm{OR}$上の次のような論理式を考える.
     \begin{equation*}
          \phi(a) := \forall (x^2 + ax + a + 3 > 0)
     \end{equation*}

     簡単のため,実閉体のモデルである実数体$\R$をとって考える.すなわち,上記の論理式は次のように言い換えられる.
     \begin{equation*}
          \text{全ての実数$x \in \R$に対して} x^2 + ax + a + 3 > 0
     \end{equation*}
     これを,変数記号$a$についての条件と解釈すると,2次方程式$x^2 + ax + a + 3 = 0$の判別式が0より大きいことに必要十分であるから,
     \begin{equation*}
          \psi(a) := (a < -2) \lor (6 < a)
     \end{equation*}
     に同値である.すなわち,次が成り立つ.
     \begin{equation*}
          \R \models \forall a (\phi(a) \leftrightarrow \psi(a)).
     \end{equation*}
\end{example}

上記の例は,実数体$\R$についてのみ考察しているが,実は,全ての$\RCF$のモデル$R$に対して
\begin{equation*}
     R \models \forall a (\phi(a) \leftrightarrow \psi(a))
\end{equation*}
が成り立つ.ここで,$\phi$には束縛変数$x$が現れており,一方で$\psi$に登場する変数記号は全て自由変数である.
つまり,$\phi$に登場する量化記号を消去し,自由変数のみの論理式$\psi$で記述できている.このような現象を扱うために,「量化記号を消去する」概念を定義する.


\begin{definition}
$T$を言語$L$上の理論とし,$\phi$を$L$上の論理式とする.% semantic consequences じゃないの? vs syntax consequences
全ての$T$のモデル$\M$に対して$\M \models \phi$が成り立つとき,$\phi$は$T$の論理的帰結(logical consequences)であるといい,$T \models \phi$とかく.

理論$T$で量化記号消去(quantifier elimination)ができるとは,
$L$上の任意の論理式$\phi(x_1,\dots,x_n)$に対し,量化記号を含まない論理式$\psi(x_1,\dots, x_n)$が存在し,
\begin{equation*}
     T \models \forall x_1 \dots \forall x_n(\phi(x_1,\dots,x_n) \leftrightarrow \psi(x_1, \dots, x_n))
\end{equation*}
が成り立つことと定義する.
\end{definition}


実は,実閉体の理論は量化記号消去ができることが示されている.すなわち,次が成り立つ.

\begin{theorem}[Tarski]\label{theorem:Tarski}
     実閉体の理論$\RCF$は量化記号消去ができる.
\end{theorem}

この定理の証明および後々の議論のため,言語$L_\mathrm{OR} = \{<, +, \cdot, 1, 0\}$上の原子論理式について注意をしておく.

まず,$1$以上の自然数の記号$n$を,$1$の$n$個の和$(1 + (1 + \dots + 1 ))$の省略記号であるとする.

さらに,$x_1, \dots, x_n$を変数記号として,多項式$f \in \Z[x_1, \dots, x_n]$が,$a_1, \dots, a_k, b_1, \dots, b_l$を$1$以上の自然数として,
\[
     f(x_1, \dots, x_n) = \sum_{i=1}^k a_i x_1^{\alpha_{i,1}} \cdots x_n^{\alpha_{i,n}} - \sum_{j=1}^l b_j x_1^{\beta_{j,1}} \cdots x_n^{\beta_{j,n}}
\]
で与えられているとする.

このとき,以下のように原子論理式の省略記号を導入する.
\begin{align*}
     \sign (f(x_1, \dots, x_n)) = 1 &\defiff \sum_{i=1}^k a_i x_1^{\alpha_{i,1}} \cdots x_n^{\alpha_{i,n}} > \sum_{j=1}^l b_j x_1^{\beta_{j,1}} \cdots x_n^{\beta_{j,n}}\\
     \sign (f(x_1, \dots, x_n)) = 0 &\defiff \sum_{i=1}^k a_i x_1^{\alpha_{i,1}} \cdots x_n^{\alpha_{i,n}} = \sum_{j=1}^l b_j x_1^{\beta_{j,1}} \cdots x_n^{\beta_{j,n}}\\
     \sign (f(x_1, \dots, x_n)) = -1 &\defiff \sum_{i=1}^k a_i x_1^{\alpha_{i,1}} \cdots x_n^{\alpha_{i,n}} < \sum_{j=1}^l b_j x_1^{\beta_{j,1}} \cdots x_n^{\beta_{j,n}}
\end{align*}

このとき,理論$\RCF$において,任意の原子論理式$\phi(x_1, \dots, x_n)$は,ある多項式$f \in \Z[x_1, \dots, x_n]$及び$\sigma \in \{-1, 0, 1\}$を用いて,
$\sign(f(x_1, \dots, x_n)) = \sigma$と同値であることに注意しておく.
また,$P = \{-1, 0, 1\}$としておく.

それでは,\cref{theorem:Tarski}を示していく.そのために,まずは次の補題を準備する.

\begin{lemma}\label{lemma:qe_simplify}
     言語$L_\mathrm{OR}$上の理論$\RCF$において,次の主張が成り立てば.理論$\RCF$は量化記号消去できる.

     \begin{enumerate}
          \item \label{qe_1}
          任意の$f_1, \dots, f_s \in \Z[X, Y_1, \dots, Y_n]$および$\map{\sigma}{\{1,\dots, s\}}{P}$に対して,量化記号のない論理式$\theta(Y_1, \dots, Y_n)$が存在し,
          \[
               \RCF \models \forall Y_1 \dots \forall Y_n ( \theta(Y_1, \dots, Y_n) \leftrightarrow \exists X(\bigwedge_i (\sign(f_i(X,Y_1, \dots, Y_n))= \sigma(i))) )
          \]
     \end{enumerate}
     を満たす.
\end{lemma}

\begin{proof}
     まず,\ref{qe_1}により次の主張が成り立つことを示す.
     \begin{enumerate}
          \setcounter{enumi}{1}
          \item \label{qe_2}
          任意の量化記号のない論理式$\phi(X, Y_1, \dots, Y_n)$に対し,ある量化記号のない論理式$\theta(Y_1, \dots, Y_n)$が存在し,
          \[
               \RCF \models \forall Y_1 \dots \forall Y_n(\theta(Y_1, \dots, Y_n) \leftrightarrow \exists X \phi(X, Y_1, \dots, Y_n))
          \]
          をみたす.
     \end{enumerate}

     まずは\ref{qe_1}より\ref{qe_2}を示す.
     任意に与えられた量化記号のない論理式$\phi(X, Y_1, \dots, Y_n)$は,多項式$f_{i,j} \in \Z[X, Y_1, \dots, Y_n]$($i=1, \dots, k, j=1, \dots, l_i$)
     と$\map{\sigma_i}{\{1, \dots, l_i\}}{P}$を用いて,和積標準形
     \[
          \phi(X, Y_1, \dots, Y_n) \leftrightarrow \bigvee_i \bigwedge_j (\sign(f_{i,j}(X, Y_1, \dots, Y_n)) = \sigma_i(j))
     \]
     で表すことができる.このとき,
     \[
          \exists X \phi(X, Y_1, \dots, Y_n) \leftrightarrow \bigvee_i (\exists X \bigwedge_j (\sign(f_{i,j}(X, Y_1, \dots, Y_n)) = \sigma_i(j)))
     \]
     が成り立つ.ここで,\ref{qe_1}より,各$i=1, \dots, k$に対して,
     \[
          \exists X \bigwedge_j (\sign(f_{i,j}(X, Y_1, \dots, Y_n)) = \sigma_i(j)) \leftrightarrow \theta_i(Y_1, \dots, Y_n)
     \]
     を満たす量化記号のない論理式$\theta_i(Y_1, \dots, Y_n)$が存在する.したがって,\ref{qe_2}は成り立つ.

     次に,\ref{qe_2}より,理論$\RCF$が量化記号消去できることを示す.
     任意の論理式$\phi$に対し,それと同値となるような量化記号のない論理式$\theta$が存在することを示せばよい.
     これは,長さに関する帰納法で示す.まず,論理式$\phi$の量化記号が0個,すなわち原子論理式の場合は,すでに量化記号消去できているのでよい.
     次に,論理記号が$n$個以下の場合に成り立っていると仮定して,論理式$\phi$の量化記号が$(n+1)$個の場合に示す.

     \begin{itemize}
          \item $\phi = \lnot \psi$の場合,$\psi$と同値な量化記号のない論理式$\theta$が存在するので,$\phi$は$\lnot \theta$と同値である.
          \item $\phi$が$\psi_1 \land \psi_2$, $\psi_1 \lor \psi_2$, $\psi_1 \rightarrow \psi_2$のいずれかの場合,否定の場合と同様にして示すことができる.
          \item $\phi = \exists X \psi$の場合,帰納法の仮定より,$\psi$は量化記号のない論理式$\theta_1$に書き直すことができる.
          よって,主張\ref{qe_2}より,$\exists X \theta_1$は量化記号のない論理式$\theta_2$に同値である.よって,示された.
          \item $\phi = \forall X \psi$の場合,まずこの論理式は$\lnot \exists X \lnot \psi$に同値であり,
          帰納法の仮定より$\psi$は量化記号のない論理式$\theta_1$に書き直すことができる.
          また,主張\ref{qe_2}より,$\exists X \lnot \theta_1$は量化記号のない論理式$\theta_2$に書き直すことができる.
          このとき,$\phi$は$\lnot \theta_2$に同値となり,示された.
     \end{itemize}     

     以上より,証明が完了した.
\end{proof}

この補題により,理論$\RCF$の量化記号消去の問題は,
整数係数の多項式の等式,および不等式の論理和の場合に,
一つの存在量化記号$\exists$を消去できればいいことに帰着した.

さらに,次のような省略記号を用意する.
まず,$1$以上の自然数$s, m$に対して,$P_{s,m}$を,$P$成分の$s$行$(2l+1)$列行列($0 \leq l \leq sm$)全体を表すとする.

$X$についての最高次数が高々$m$である多項式列$f_1, \dots, f_s \in \Z[Y_1, \dots, Y_n][X]$と,
$s$行$(2N+1)$列行列$A = (a_{i,j}) \in P_{s, m}$($0 \leq N \leq sm$)について,
\[
     \SIGN(f_1, \dots, f_s) = A
\]
を,以下の事を書き表した論理式の省略であるとする.($A$を,$f_1, \dots, f_s$の符号表という.)
また,この論理式の自由変数は高々$Y_1, \dots, Y_n$のみである.

ある$x_1<\dots<x_N$が存在して,
\begin{enumerate}
    \item $\{x_1, \dots, x_N\} = \{x \mid \text{$x$は$\Z[Y_1, \dots, Y_n]$に属さないようなある$f_i$の解}\}$,
    \item $x_0 := -\infty$, $x_{N+1} := \infty$として,区間$I_k = (x_k, x_{k+1}) (0 \leq k \leq N)$と$p \in P$に対して,($I_k$上では各$f_i$は定符号であるので\footnote{この事実は,実閉体の場合の中間値の定理を用いている.})
    \[
          \sign(f_i(I_k))=p \defiff \exists x (x \in I_k \land \sign(f_i(x) = p))
    \]
    とおく.このとき,どんな$1 \leq i \leq s$と,どんな$0 \leq k \leq N$に対しても
    \[
          \sign(f_i(I_k))=a_{i,2k+1} \land (k>0 \rightarrow \sign(f_i(X_k))=a_{i,2k})
    \]
    である.
\end{enumerate}

このとき,次が成り立つ.
\begin{lemma}\label{lemma:qe_1ststep}
     与えられた$\map{\sigma}{\{1, \dots, t\}}{P}$に対して,部分集合$P(\sigma) \subset P_{s,m}$を,次を満たすように取れる.
     最高次数は高々$m$である任意の多項式列$f_1, \dots, f_s \in \Z[Y_1, \dots, Y_n][X]$に対し,
     \[
          \SIGN(f_1, \dots, f_s) \in P(\sigma) \leftrightarrow \exists X(\bigwedge_i(\sign(f_i(X, Y_1, \dots, Y_n)) = \sigma(i)))
     \]

     ここで,$\SIGN(f_1, \dots, f_s) \in P(\sigma)$は,論理式
     \[
          \bigvee_{A \in P(\sigma)} (\SIGN(f_1, \dots, f_s) = A)
     \]
     の省略である.
\end{lemma}

\begin{proof}
     \[
          P(\sigma):=\{A \in P_{s,m} \mid \text{行列$A$のある列が$(\sigma(i))_{i=1}^s$である}\}
     \]
     とすればよい.
\end{proof}

% ここで,多項式の符号表を微分により次数下げを行った多項式の符号表により計算するアルゴリズムを導入する.
% 符号表に関する条件が最終的に定数の符号表に関する条件に帰着され,量化記号のない論理式に同地であることが示せる.

% 最後に,\cref{theorem:Tarski}を示すための重要な仕組みである二つのアルゴリズムを導入する.

% 一つは,長さが$s$である多項式の列$f_1, \dots, f_s \in \Z[X, Y_1, \dots, Y_n]$に対して,
% 長さが$2s$である多項式の列$\calS(f_1, \dots, f_s) = (g_1, \dots, g_{2s})$を返すアルゴリズム$\calS$である.

% もう一つが,行列$A \in P_s,m$及び$s列$のベクトル(n_1, \dots, )

以上の準備をもとに,\cref{theorem:Tarski}を示す.

\begin{proof}[\cref{theorem:Tarski}の証明]
     まず,\cref{lemma:qe_simplify}により,任意に与えられた$f_1, \dots, f_s \in \Z[X, Y_1, \dots, Y_n]$および$\map{\sigma}{\{1, \dots, s\}}{P}$に対して,
     量化記号のない論理式$\theta(Y_1, \dots, Y_n)$が存在し,
     \[
          \RCF \models \theta(Y_1, \dots, Y_n) \leftrightarrow \exists X(\bigwedge_i(\sign(f_i(X,Y_1, \dots, Y_n)) = \sigma(i)))
     \]
     を満たせばよい.ここで,\cref{lemma:qe_1ststep}より,
     \[
          \RCF \models \SIGN(f_1, \dots, f_s) \in P(\sigma ) \leftrightarrow \exists X(\bigwedge_i(\sign(f_i(X,Y_1, \dots, Y_n)) = \sigma(i)))
     \]
     が成り立つ.

     % まだ証明の途中
\end{proof}


% しかし,上記の定理は,任意の論理式が与えられたとき,それに同値な量化記号を含まない論理式を導出する手続きを示しているわけではない.
% よりアルゴリズミックに導出する手続きとして,Collinsにより提唱された柱状代数分解による量化記号消去の方法が知られている.

% この方法は,次のような形式で与えられた論理式
% \begin{equation*}
%      \Qua_1 y_1 \Qua_2 y_2 \dots \Qua_k y_k \psi(x_1, \dots, x_n, y_1, \dots, y_k)
% \end{equation*}
% に対して同値な$量化記号を含まない論理式を求める手続きを与えている.
% ここで,$\Qua_i \in \{\exists, \forall\}$であり, $\psi(x_1, \dots, x_n, y_1, \dots, y_k)$は量化記号を含まない論理式である. 

% 次の節では,その柱状代数分解について解説する.

\section{柱状代数分解}

$R$を実閉体とする.

集合$X$の部分集合族$\calS$が,
\begin{enumerate}
     \item $X = \bigcup_{S \in \calS} S$,
     \item 任意の$S_1, S_2 \in \calS$に対し,$S_1 \neq S_2$ならば$S_1 \cap S_2 = \emptyset$.
\end{enumerate}
を満たすとき,部分集合族$\calS$を$X$の分解(decomposition)と呼ぶ.

まず始めに,$R^n$の柱状代数分解を定義する.この定義は,参考文献\cite{Basu}による.

\begin{definition} \label{definition:cad}
     $i=1, \dots, n$に対し,$\calS_i$を$R^i$の分解とする.
     $\{\calS_i\}_{i=1}^n$が$R^n$の柱状代数分解(cylindrical algebraic decomposition)であるとは,
     以下の条件を満たすことをいう.
     \begin{enumerate}
          \item $S \in \calS_1$は,$R$上の点か,開区間かのいずれかである.
          \item $n\geq 2$の場合,任意の$i=1, \dots, k-1$と任意の$S \in \calS_i$に対し,
          有限個の連続な半代数的関数
          \[
               \map{\xi_{S,1}< \dots <\xi_{S,l_S}}{S}{R}
          \]
          が存在し,以下を満たす.(ただし,$l_S$は0以上の自然数とする.)
          \begin{itemize}
               \item 各$j=1 \dots, l_S$に対し,$\xi_{S,j}$のグラフ
               \[
                    \{(x,x_{i+1}) \mid x \in S, \xi_{S,j}(x)=x_{i+1} \} \subset R^{i+1}
               \]
               は,$\calS_{i+1}$の元である.
               \item $\xi_{S,0}=-\infty$, $\xi_{S,l_S+1}=\infty$とするとき,各$j=0, \dots, l_S$に対し,\label{cad_condition1}
               \[
                    \{(x,x_{i+1}) \mid x \in S, \xi_{S,j}(x)<x_{i+1}<\xi_{S,j+1} \} \subset R^{i+1}
               \]
               は,$\calS_{i+1}$の元である.
          \end{itemize}
     \end{enumerate}
\end{definition}

定義より,任意の$k=1, \dots, n$に対して,$\{\calS_i\}_{i=1}^k$は$R^k$の柱状代数分解である.
逆に,$R^{n-1}$の柱状代数分解$\{\calS_i\}_{i=1}^{n-1}$が与えられているとき,各$S \in \calS_{n-1}$に対して
\cref{definition:cad}の条件\ref{cad_condition1}を満たすような有限個の連続半代数関数$\xi_{S,1}<\dots<\xi_{S,l_S}$が与えられれば,
$R^n$の柱状代数分解$\{\calS_i\}_{i=1}^n$を構成することができる.
このように,柱状代数分解は再帰的な構造である.

\begin{definition}
     $F \subset R[X_1, \dots, x_n]$を有限部分集合とする.
     \begin{enumerate}
          \item 部分集合$S \subset \R^n$が$F$符号不変($F$-invariant)であるとは,
          任意の$f \in F$及び任意の$x,y \in S$に対し$\sign(f(x))=\sign(f(y))$であることと定義する.
          \item $R^n$の柱状代数分解$\{\calS_i\}_{i=1}^n$が$F$に適合している(adapted to $F$)とは,
          任意の$C \in \calS_k$が$F$符号不変であることと定義する.
     \end{enumerate}
\end{definition}

この節の目的は,次の定理を示すことである.

\begin{theorem} \label{theorem:cad}
     任意の有限部分集合$F \subset R[X_1, \dots, x_n]$に対して,
     $F$に適合した$R^n$の柱状代数分解が存在する.
\end{theorem}

この定理は,柱状代数分解を再帰的に構成することで証明される.
まず初めに$R^1$の柱状代数分解$\{\calS_1\}$を与え,
それを基に順に$R^k$の柱状代数分解$\{\calS_i\}_{i=1}^k$を構成していく.(ただし$k=2, \dots, n$.)

問題になるのは,$F$に適合した$R^n$の柱状代数分解を構成することである.
以下の節では,各$k=1, \dots, n-1$で$R^k$の柱状代数分解$\{\calS_i\}_{i=1}^k$を構成するにあたり
どのような条件を満たす必要があるかを述べていく.

\subsection{描画可能}
$F$に適合した$R^n$の柱状代数分解$\{\calS_i\}_{i=1}^n$を$R^{n-1}$の柱状代数分解$\{\calS_i\}_{i=1}^n$から構成するとき,
各$S \in \calS_{n-1}$は次に定義する性質を満たしてほしい.

なお,次の定義は参考文献\cite{Collins}によって導入されたものである.

\begin{definition} 
     $F \subset R^n[X_1, \dots, X_n]$を有限部分集合とし,$S \subset R^{n-1}$を空でない半代数的連結な半代数的集合とする.
     $S$が$F$描画可能($F$-delineable)であるとは,$k$個の連続な半代数的関数$\map{\xi_1<\dots<\xi_k}{S}{R}$が存在し,次を満たすことと定義する.
     \begin{itemize}
          \item 任意の$ x \in S $に対し,
          \[
               \{\xi_i(x)\}_{i=1}^k = \{y \in R \mid \prod_{f \in F'}f(x,y)=0\}
          \]
          である.ただし,$F' = \{f \in F \mid \text{$f$は$S$上恒等的に0でない}\}$とする.
          \item 各$i=1, \dots, k$および$f \in F'$に対し,多項式$f(x,X_n)$における根$\xi_i(x)$の重複度は,$x\in S$によらず一定.
     \end{itemize}
\end{definition}

$R^{n-1}$の柱状代数分解$\{\calS_i\}_{i=1}^n$で,
各$S \in \calS_{n-1}$が$F$描画可能であるとき,定義より得られる連続な半代数関数$\map{\xi_1<\dots<\xi_k}{S}{R}$を
各$S$に対して取ることにより,$F$適合な$R^n$の柱状代数分解$\{\calS_i\}_{i=1}^n$を得ることができる.

$F$描画可能であることの必要条件は,次のように与えられる.

\begin{proposition}\label{proposition:del}
     $F \subset R[X_1, \dots, X_n]$を有限部分集合とし,$S \subset \R^{n-1}$を半代数的連結な半代数的集合とする.
     さらに,以下を満たすとする.
     \begin{enumerate}
          \item 任意の$f \in F$に対し,$S$上$f$の$R[\sqrt{-1}]$上の根の数は重複度込みで一定である.
          \item 任意の$f \in F$に対し,$S$上$f$の$R[\sqrt{-1}]$上の相異なる根の数は一定である.
          \item 任意の$f, g \in F$に対し,$S$上$f$, $g$に共通する$R[\sqrt{-1}]$上の根の数は重複度込みで一定である.
     \end{enumerate}
     このとき,$S$は$F$描画可能である.
\end{proposition}

\cref{proposition:del}を示すために,次の二つの補題を用意する.

\begin{lemma}\label{lemma:del_1}
     $S \subset R^{n-1}$を空でない半代数的連結な半代数的集合部分集合とし,
     $f_1, f_2 \in R[x_1, \dots, x_n]$を$S$上恒等的に$0$でない多項式とする.
     $f_1, f_2$が次を満たすとする.
     \begin{enumerate}
          \item 各$i=1, 2$に対し,$S$上$f_i$の$R[\sqrt{-1}]$上の根の数は重複度込みで一定.
          \item 各$i=1, 2$に対し,$S$上$f_i$の$R[\sqrt{-1}]$上の相異なる根の数は一定.
          \item $S$上$f_1, f_2$の$R[\sqrt{-1}]$上の共通根の数は重複度込みで一定.
     \end{enumerate}
     このとき,$S$上$f_1, f_2$の相異なる$R[\sqrt{-1}]$上の共通根の数は一定である.
\end{lemma}
\begin{proof}
     簡単のため,$Y:=X_n$, $X:=(X_1, \dots, X_{n-1})$と表す.
     このとき,$f_1, f_2$は$f_1(X,Y), f_2(X,Y)$と表すことができる.
     $x \in S$に対して,$f_1(x,Y), f_2(x,Y)$の$R[\sqrt{-1}]$上の相異なる共通根の数を$N(x)$と定義する.
     すると,$\map{N}{S}{R}$は半代数的関数である.
     \footnote{$N$が半代数的関数であることは,$N$のグラフを以下のように書き表すことができるためである:
     \[
          \{(x,z) \mid \bigvee_{n}(z = n \text{ and } \exists y_1, \dots, y_n \text{ s.t. } y_i \neq y_j( i\neq j) \text{ and } \forall y(f_1(x,y)=f_2(x,y) \leftrightarrow \bigvee_{i=1}^n (y = y_i)))\}
     \]
     ここで,論理和の$n$は,$0$から$gcd(P,Q)$の$Y$に関する次数までをとる.
     この主張はすでに$\RCF$において$量化記号消去$が可能であることを用いてしまっているが,
     それは別の方法で示すことができる.(例えば新井\cite{Arai}を参照.)
     }

     次の主張を示せば,$N$は$S$上の局所定数半代数的関数であるため,$N$は定数関数となり,証明が完了する.
     \begin{claim*}
          任意の$x \in S$に対し,$x$の開近傍$V$で,
          $S\cap V$上$f_1(X,Y), f_2(X,Y)$の$R[\sqrt{-1}]$上の相異なる共通根の数が一定となるようなものが存在する.
     \end{claim*}
     以降は主張を示す.
% メモ:
% この後は,方程式の解が係数に関して連続であることから示される.
%(実閉体の場合も同様に示すことができる,証明は逆関数定理を経由する.)
\end{proof}

\begin{lemma}\label{lemma:del_2}
     $S \subset R^{n-1}$を半代数的連結な半代数的集合とし,     
     $A \in R[x_1, \dots, x_n]$を$S$上恒等的に$0$でない多項式とする.
     さらに,次を満たすとする.
     \begin{itemize}
          \item $S$上$A$の$R[\sqrt{-1}]$上の根の数は重複度込みで一定.
          \item $S$上$A$の$R[\sqrt{-1}]$上の相異なる根の数は一定.
     \end{itemize}
     このとき,$S$は$\{A\}$描画可能である.
\end{lemma}

\begin{proof}
     
\end{proof}

% ここからは実閉体上で書いているため,チェックをしておく.

% \begin{lemma}\label{lemma:del_2}
% $A \in \R[x_1, \dots, x_n]$とし,$S \subset \R^{n-1}$を弧状連結部分集合とする.
% \begin{itemize}
% \item $S$上$A$の重複度込みの複素数根の数は一定.
% \item $S$上$A$の相異なる複素数根の数は一定.
% \end{itemize}
% このとき,$S$は$\{A\}$描画可能である.
% \end{lemma}

% \begin{proof}
% 方針:次の二つのことを示さなければならない.\\
% 1.  $S$上実根の数は一定.\\
% 2.  $S$上実根は連続である.\\
% いずれも多項式の根の係数に対する連続性から示せる.
% \end{proof}

\begin{proof}[{\bf \cref{proposition:del}の証明}]
任意の$f \in F$は$S$上恒等的に$0$でないとしてよい.

\cref{lemma:del_2}より,各$f \in F$に対して,$S$は$\{f\}$描画可能である.
よって,$S$上の連続関数$\alpha_{1,f}(a) < \dots \alpha_{n_f, f}(a)$を,各$a \in S$で$f(a)(x) \in \R[x]$の解であるようにとれる.

次の主張が成り立てばよい.

\begin{claim*}
$f, g \in F$が,$f \neq g$であるとする.
ある$a \in S$において$\alpha_{k,f}(a) = \alpha_{l,g}(a)$であるならば,任意の$a \in S$に対して$\alpha_{k,f}(a) = \alpha_{l,g}(a)$である.
\end{claim*}

この主張は,$S$が半代数的連結であることと,\cref{lemma:del_1}から従う.よって,命題が示された.
\end{proof}

\begin{corollary}\label{corollary:del}
$S \subset \R^{n-1}$を弧状連結部分集合とし,$F \subset \R[x_1,\dots, x_n]$を有限部分集合とする.
次が成り立つとき,$S$は$F$描画可能である.
\begin{itemize}
\item 任意の$f \in F$に対し,$\deg(f(a))$が一定($a \in S$).
\item 任意の$f \in F$に対し,$\deg(\gcd(f(a), \frac{\partial f}{\partial x_n}(a)))$が一定($a \in S$).
\item 任意の$f, g \in F$に対し,$\deg(\gcd(f(a), g(a)))$が一定($a \in S$).
\end{itemize}
\end{corollary}

よって,$F$符号不変な$\R^n$の分割を与えるには,\cref{corollary:del}の条件を満たすような$\R^{n-1}$の分割を構成すればよい.
一つ目の条件を満たすような$\R^{n-1}$の分割を与えるには,各$f \in F$について,$x_n$係数が符号不変になるような分割を構成すればよい.
しかし,二つ目と三つ目の条件を満たすような$\R^{n-1}$の分割を与えるのは少し難しい.
なぜなら,多項式$f, g \in \R[x_1, \dots, x_n]$について,$a \in \R^{n-1}$を固定したとき,$\gcd(f,g)(a)$と$\gcd(f(a),g(a))$は必ずしも等しくないからである.

よって,二つ目と三つ目の条件も満たすような$\R^{n-1}$の分割を与えることができるように次の節で準備をする.

\subsection{主部分終結式係数(Principal Subresultant Coefficient)}

\begin{definition}
$\mathrm{R}$を可換環とし,$A(x), B(x) \in \mathrm{R}[x]$ を,$\deg A(x) = m$, $\deg B(x) = n$ とする.ただし,$\deg 0 = 0$と解釈する.

$j = 0, \dots, \min\{n, m\}$に対し,多項式$A(x)$, $B(x)$の$j$次部分終結式$S_j(A, B)$を次のように定義する.
\begin{align*}
A(x) = a_m x^m + \dots + a_1 x + a_0, \\
B(x) = b_n x^n + \dots + b_1 x + b_0 
\end{align*}
として,$j = 0, \dots, \min\{n,m\}$に対し,
\begin{align*}
M_j = 
\begin{pmatrix}
a_m & a_{m-1} & \cdots & a_1 & a_0 &    &  \\
     &  a_m     & \cdots & a_2 & a_1& a_0 &  \\
     &   & \ddots &  & & \\
b_n & b_{n-1} & \cdots & b_1 & b_0 &    & \\
     &  b_n     & \cdots & b_2 & b_1& b_0 & \\
     &   & \ddots &  & & 
\end{pmatrix}
\in \mathrm{M}_{m+n-2j, m+n-j}(\mathrm{R})
\end{align*}
とする.ここで,行列の空白部分はすべて$0$であり,また,行列の上側は$n-j$行,行列の下側は$m-j$行である.

また,$j = 0, \dots, \min\{m,n\}$, $i = 0, \dots, j$に対し,
\begin{align*}
M_{j,i} = (\text{$M_j$の第$1$列}, \text{$M_j$の第$2$列}, \dots ,\text{$M_j$の第$m+n-2j-1$列}, \text{$M_j$の第$m+n-i-j$列})
\in \mathrm{M}_{m+n-2j, m+n-2j}(\mathrm{R})
\end{align*}
とする.$j = 0, \dots, \min\{n, m\}$に対し,
\begin{align*}
S_j(A, B) = \sum_{i=0}^j \det M_{j, i} \cdot x^i 
\end{align*}
を多項式$A(x)$, $B(x)$の$j$次部分終結式$S_j(A, B)$という.

この$j$次部分終結式の先頭項係数を,$\psc_j(A,B)$とかき,多項式$A(x)$, $B(x)$の$j$次主部分終結式係数という.
\end{definition}


主部分終結式係数は,多項式$A(x)$, $B(x)$の係数,及び次数に依存して定まる.

\begin{remark}
$m$, $n$のどちらかが$0$のとき,
$S_0(A,B) = 0$, $\psc_0(A,B) = 0$とする.

また,$\psc_0(A,B)$は,多項式$A(x)$, $B(x)$の終結式に一致する.
\end{remark}

\begin{proposition}\label{proposition:psc}
$\mathrm{R}$を体とし,$A(x), B(x) \in \mathrm{R}[x] \setminus \{0\}$とすると,次が成立する.
\begin{align*}
\deg(\gcd(A, B)) = \min \{ j  \in \{0,1, \dots, \min\{n,m\}\}\mid \psc_j(A,B) \neq 0\}
\end{align*}
\end{proposition}

\begin{proof}
後で書く.
方針: 部分終結式がユークリッドの互除法で出てくる多項式の列の定数倍になることが分かる.
\end{proof}

\subsection{符号不変な分割の存在}
ここでは\cref{theorem:cad}を証明する.
そのために,与えられた有限部分集合$F \subset R[X_1, \dots, X_n]$に対して,
帰納的な議論で$R^n$の$F$適合な柱状代数分解の存在を示す.

まず,帰納的な議論のために,次のような記号を定義する.

\begin{definition}
$F \subset R[X_1, \dots, X_n]$を有限部分集合とする.ただし,$n \geq 2$とする.
$\PROJ(F) \subset R[X_1, \dots, X_{n-1}]$を,次のように定める.

まず,$B(F) := \{ \mathrm{red}^k(f) \mid f \in F, k=1, \dots, \deg(f) \}$とする.ただし,$\mathrm{red}(f) = f - \mathrm{LT}(f; X_n)$である.
次に,
\begin{align*}
	\PROJ_1(F) &:= \{\mathrm{LC}(f;X_n) \mid f \in B\},\\
	\PROJ_2(F) &:= \{\psc_j(f, \frac{\partial f}{\partial X_n}; X_n) \mid f \in B, j =0, \dots, \deg(\frac{\partial f}{\partial X_n};X_n) \},\\
	\PROJ_3(F) &:= \{\psc_j(f,g;X_n) \mid f,g \in B, j = 0, \dots, \min\{\deg(f;X_n), \deg(g;X_n)\}\}
\end{align*}
とし,以上を用いて
\begin{align*}
	\PROJ(F) &:= \PROJ_1(F) \cup \PROJ_2(F) \cup \PROJ_3(F)
\end{align*}
と定める.
\end{definition}

この記号のもとで,次の系が成り立つ.

\begin{corollary}\label{corollary:inv-deline}
$F \subset \R[X_1, \dots, X_n]$を有限部分集合とし,$S \subset \R^{n-1}$を半代数連結な半代数的集合とする.

$S$が$\PROJ(F)$符号不変ならば,$S$は$F$描画可能である.
\end{corollary}

\begin{proof}
\cref{proposition:del}及び\cref{proposition:psc}から従う.
\end{proof}

それでは,\cref{definition:cad}の証明を述べる.

\begin{proof}[\cref{theorem:cad}の証明]
     $n$についての帰納法によって示す.

     $n=1$の場合,$F$の$R$上の根を$a_1 < \dots, a_k$とする.
     $a_0 = -\infty$, $a_{k+1} = \infty$とすれば,
     \[
          \calS_1 = \{(a_i, a_{i+1})\}_{i=0}^k \cup \{a_i\}_{i=1}^k
     \]
     が$F$-適合な柱状代数分解を与える.よって,定理は成り立つ.

     $n\geq 2$の場合,$n-1$までで主張が成り立つとする.
     このとき,帰納法の仮定により$\PROJ(F)$適合な$R^{n-1}$の柱状代数分解を$\{\calS_i\}_{i=1}^{n-1}$がとれる.

     ここで,各$S \in \calS_{n-1}$は$\PROJ(F)$不変であるから,\cref{corollary:inv-deline}より,$S$は$F$描画可能である.

     従って,各$S \in \calS_{n-1}$に対して,有限個の半代数的連続関数$\map{\xi_{S,1}< \dots < \xi_{S,l_S}}{S}{R}$が存在し,
     \begin{itemize}
          \item 任意の$ x \in S $に対し,
          \[
               \{\xi_{S,i}\}_{i=1}^{l_S} = \{y \in R \mid \prod_{f \in F'}f(x,y)=0\}
          \]
          である.ただし,$F' = \{f \in F \mid \text{$f$は$S$上恒等的に0でない}\}$とする.
          \item 各$i=1, \dots, l_S$および$f \in F'$に対し,多項式$f(x,X_n)$における根$\xi_{S,i}(x)$の重複度は,$x\in S$によらず一定.
     \end{itemize}
     であるようにとれる.
     また,$\xi_{S,0} := -\infty$, $\xi_{S,l_S} := \infty$としておく.

     このとき,
     \begin{align*}
          C_{S,2i} &:= \{(x,y) \mid  x \in S, \xi_{S,i}(x) = y \} \quad i = 1,\dots, l_S,\\
          C_{S,2i+1} &:= \{(x,y) \mid x \in S, \xi_{S,i}(x)<y<\xi_{S,i+1}(x) \} \quad i = 0,1, \dots, l_S 
     \end{align*}
     とし,
     \[
          \calS_n := \{C_{S,i} \mid S \in \calS_{n-1}, i=1, \dots, 2l_S+1\}
     \]
     とすれば,$\{\calS_i\}_{i=1}^n$は,$R^n$の$F$適合な柱状代数分解を与える.

     よって,帰納法により示された.
\end{proof}

% \begin{definition}
% $ F \subset \mathbb{R}[x_1,\dots,x_n] $ を有限部分集合とする.$ S \subset \mathbb{R}^{n-1} $が$ F $描画可能とする.
% このとき,$ S $上の$ F $の解を$ f_1(x)< \dots <f_k(x) $とし,$ f_0(x) := -\infty $ , $ f_{k+1}(x) := \infty $とするとき,
% \begin{align*}
%   C_{2i} &:= \{(x,y) \mid  x \in S, f_i(x) = y \} \quad i = 1,\dots, k,\\
%   C_{2i+1} &:= \{(x,y) \mid x \in S, f_{i}(x)<y<f_{i+1}(x) \} \quad i = 0,1, \dots, k 
% \end{align*}

% とすれば,$\{C_j\}_{j=1}^{2k+1}$は$ S \times \mathbb{R} $の$F$符号不変な分割を与える.
% この$ S \times \mathbb{R} $の分割$ \{C_j\}_{j=1}^{2k+1} $を,$ S $の持ち上げといい,$ \mathfrak{L}(S) $と書く.

% また,$\mathfrak{D}$が$\R^{n-1}$の分割であるとき,$\mathfrak{L}(\mathfrak{D}) := \bigcup_{D \in \mathfrak{D}}\mathfrak{L}(D)$を$\mathfrak{D}$の持ち上げという.
% \end{definition}

\section{柱状代数分解による量化記号消去}
有限部分集合$F \subset R[x_1, \dots, x_n]$が与えられたとき,
前節では$\R^n$の$F$適合な柱状代数分解$\mathfrak{D}$が存在することを示した.
この節では,柱状代数分解による量化記号消去のアルゴリズムについて示す.

冠頭標準形の論理式
\[
     \Qua_1 X_1 \dots \Qua_1 X_k \phi(Y_1, \dots, Y_l, X_1, \dots, X_k)
\]
が与えられたとき,
$R^{k+l}$の柱状代数分解$\{\calS_i\}_{i=1}^{k+l}$を用いて量化記号消去を行うことを考える.

そのためには,$\calS_l$に属する各$S \in \calS_l$の定義式が与えられている必要がある.
しかし,前節で示した柱状代数分解の構成では,各$S \in \calS_l$の定義式は与えられない.

そのため,柱状代数分解のアルゴリズムを改良し,
$\{\calS_i\}_{i=1}^l$に属する各$S \in \calS_i$の定義式を与えるようにする必要がある.
そのために,まずは増補射影について定義する.

\subsection{増補射影}
     多項式の有限部分集合$F \subset R[X]$が,微分について閉じている(closed under differentiation)
     とは,任意の$f \in F$に対して,$f' \in F$または$f'=0$となることと定義する.

\begin{theorem}[Thomの補題]
     $F = \{f_1, \dots, f_s\} \subset R[X]$を,微分について閉じた有限部分集合とする.
     このとき, 任意の$\map{\sigma}{\{1, \dots, s\}}{\{-1,0,1\}}$に対し,
     \[
          A_\sigma := \{x \in R \mid \text{任意の$i=1, \dots, s$に対し,} \sign(f_i(x)) = \sigma(i)\}
     \]
     と定義する.
     このとき,$A_\sigma$は空集合,一点からなる集合,開区間のいずれかである.
\end{theorem}

\begin{proof}
     $s$に関する帰納法で示す.
     $s = 0$のときは明らかである.
     
     次に,$s$において主張が成り立つとする.
     $F = \{f_1, \dots, f_{s+1}\}$とする.ここで,順番を並び替えて$f_{s+1}$の次数が最大であるとする.

     任意に$\map{\sigma}{\{1, \dots, s, s+1\}}{\{-1,0,1\}}$をとる.
     $\map{\sigma_0}{\{1, \dots, s\}}{\{-1,0,1\}}$を$\sigma$の制限,すなわち
     \[
          \sigma_0(i) := \sigma(i), \quad i=1, \dots, s
     \]
     とする.

     このとき,$F_0 := \{f_1, \dots, f_{s}\}$は微分について閉じているので,帰納法の仮定より
     \[
          A_{\sigma_0}:=\{x \in R \mid \text{任意の$i=1, \dots, s$に対し,} \sign(f_i(x)) = \sigma_0(i)\}
     \]
     は,空集合,一点からなる集合,開区間のいずれかである.

     $A_\sigma = A_{\sigma_0} \cap \{x \in R \mid \sign(f_{s+1}(x)) = \sigma(s+1)\}$であるから,
     $A_{\sigma_0}$が空集合,あるいは一点からなる集合の場合には,$A_\sigma$は空集合か一点からなる集合になるので主張は成り立つ.
     したがって,$A_{\sigma_0}$が開区間である場合を考える.
     $f_{s+1}$の微分$f_{s+1}'$は,$f_1, \dots, f_s$のいずれかであるから,$A_{\sigma_0}$上符号は一定である.
     $f_{s+1}'$の符号が$A_{\sigma_0}$上常に$0$である場合,$f_{s+1}$は$A_{\sigma_0}$上定数であるから,主張は成り立つ.
     $f_{s+1}'$の符号が$A_{\sigma_0}$上常に$-1$(あるいは$1$)の場合を考える.
     このとき$f$は$A_{\sigma_0}$上狭義単調増加(あるいは競技単調減少)であるから,$A_{\sigma}$は空集合,一点からなる集合,開区間のいずれかである.

     以上より,主張は示された.
\end{proof}

\begin{definition}[増補射影]
     後で書く.

\end{definition}

増補射影を用いることで,帰納的に定義式が分かっているような分割を得ることができる.

\begin{theorem}
     $F \subset R[X_1, \dots, X_n]$を有限部分集合とし,$n \geq 2$とする.
     $S \subset R^{n-1}$を,半代数的連結な半代数的部分集合とする.
     $S$が$\APROJ(F)$不変ならば,$S$は$\der(F)$描画可能である.

     特に,$\map{\sigma}{\{1,\dots, s\}}{\{-1, 0, 1\}}$と$g_1, \dots g_s \in R[X_1, \dots, X_{n-1}]$を用いて,
     \[
          S = \{(x_1, \dots, x_{n-1}) \in R^{n-1} \mid \sign(g_i(x_1, \dots, x_{n-1}))= \sigma_i\}
     \]
     であるならば,
     $\der(F)$による$S$の持ち上げ$\calL(S)$の定義式は,$\{g_1, \dots,g_s\} \cup \der(F)$で記述できる.
\end{theorem}

\subsection{柱状代数分解による量化記号消去のアルゴリズム}

柱状代数分解による量化記号消去のアルゴリズムは,次のように述べられる.

\begin{algorithm}
     \caption{QE}
     \begin{algorithmic}[1]
     \REQUIRE 
     \ENSURE 
     \STATE pass
     \end{algorithmic}
\end{algorithm}

\begin{thebibliography}{99}
     \bibitem{Arai} 新井敏康,『数学基礎論』(2016).
     \bibitem{Itai} 板井昌典,『モデル理論』(2023).
     \bibitem{Bochnak}Jacek Bochnak, Michel Coste, Marie-Fran\c{c}oise Roy, Real Algebraic Geometry, Springer(1998).
     \bibitem{Basu} Saugata Basu, Richard Pollack, Marie-Fran\c{c}oise Roy, Algorithms in Real Algebraic Geometry(2003).
     \bibitem{Collins} George E. Collins, Quantifier elimination for real closed fields by cylindrical algebraic decomposition(1975).
     \bibitem{Anai} 穴井宏和,横山和弘,『QEの計算アルゴリズムとその応用 数式処理による最適化』(2011).
\end{thebibliography}

\end{document}