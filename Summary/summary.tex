\documentclass[uplatex, dvipdfmx]{jsarticle}

\usepackage{todonotes}
\usepackage{amsmath,amssymb,mathtools}
\usepackage{amsthm}
\usepackage{cleveref}

\newcommand{\Qua}{\mathcal{Q}}
\newcommand{\map}[3]{{#1}\colon{#2}\rightarrow{#3}}

\theoremstyle{definition}
\newtheorem*{definition*}{定義}

\begin{document}
\begin{center}
    {\LARGE \centering 論文内容の要旨}
\end{center}

\vskip\baselineskip

{\huge \underline{修士論文題目}}

\vskip\baselineskip

{\huge \underline{実閉体上の柱状代数分解による量化記号消去}}

\vskip\baselineskip

{\LARGE 氏名 \quad 富永 直弥}

\vskip\baselineskip

本論文は,実閉体の理論が量化記号消去を持つことを述べるTarskiの定理と,
Collinsによる柱状代数分解を用いた実閉体上の量化記号消去のアルゴリズムを紹介する総合報告である.

実閉体とは,ArtinとSchreier \cite{MR3069467}により導入された,実数体の代数的な性質を公理化して定義される概念である.
Tarskiは,実閉体の理論が量化記号消去を持つことを示した\cite{MR0044472}.
すなわち,実閉体の理論において,任意の論理式はある量化記号のない論理式に同値である.

Tarskiによる証明は構成的であったが,実行時間は指数関数の有限回の合成で抑えることができないことが知られている.
Collinsが,より実用的な新しい量化記号消去のアルゴリズムを提案した\cite{MR0403962}.
Collinsによる量化記号消去のアルゴリズムの計算量は$(Mkd)^{2^{O(n)}}$である\cite{MR0949113}.
ここで,量化記号消去を行う論理式は冠頭標準形であるとし,
その中に現れる多項式$f_1, \dots, f_k$は,$\max_{i=1,\dots, k}\deg(f_i) \leq d$を満たし,
さらに各$f_i$の各係数の絶対値が$2^M$で抑えられているとする.

Collinsによる量化記号消去のアルゴリズムは,柱状代数分解を用いている.
ここで,「代数分解」は,空間を非交叉な半代数的集合に分解することを意味する.
半代数的集合とは,有限個の多項式の不等式$f>0$および等式$f=0$の定める集合の直和,直積,または補集合により得られる空間である.
そして,「柱状代数分解」とは,余次元$1$の部分空間の代数分解を,柱状に拡張することで得られる代数分解である.
このような概念は,一般の実閉体において定義される.

Collinsの元論文\cite{MR0403962}では,一般の実閉体ではなく,実数体$\mathbb{R}$に限って議論を行っているが,
一般の実閉体の場合でも同様の議論が可能である.
しかし,実閉体の場合に同様の議論を行うには,実閉体や半代数的集合の性質が,
実数体$\mathbb{R}$の場合と同様の振舞いをすることを示す必要がある.

そのため,本論文の前半では実閉体および半代数的集合の性質についてまとめる.
まず,2節では,実閉体の定義を行い,その性質について述べる.
次に,3節では1階述語論理についての準備を行い,4節ではTarskiの定理の証明をする.
そして,5節では実閉体における半代数的集合の定義を行い,その性質についてまとめる.

本論文の後半では,Collinsによる量化記号消去のアルゴリズムを述べる.
Collinsによるアルゴリズムはいくつかのサブアルゴリズムに分かれているが,
その中で中心となるのは柱状代数分解を求めるアルゴリズムである.

量化記号消去においては,有限個の整数係数多項式$f_1, \dots, f_s$が与えられ,
柱状代数分解を,各$f_i$ ($i=1,\dots,s$) の符号が分解の各領域において一定であるようにとる必要がある.
6節では,そのような柱状代数分解を取る方法について述べる.

しかし,この方法では,得られた柱状代数分解の各領域が,どのような多項式で定義されているかまではわからない.
そこで,7節では6節の方法を改良することで,柱状代数分解の各領域を定める多項式までわかる方法について述べる.
そして8節で,柱状代数分解と量化記号消去の関係について述べ,
最後に9節で柱状代数分解による量化記号消去のアルゴリズムの疑似コードを提示する.


\bibliographystyle{alpha}
\bibliography{references.bib}
\end{document}